\section{November 2013 \hrulefill}

\subsection{Saturday 2nd}
Read about ionicons, which are a useful set of MIT licensed icons that will be
useful when designing the UI \cite{Sperry2013}.

\subsection{Sunday 3rd}
Began work on implementing static page designs based on D1 mockups, using
Bootstrap.

\paragraph{TODO}
\begin{itemize}
\item Homepage - DONE
\item Advanced Search - DONE
\item Search results
\item Details page
\item Login page
\item Add data
\end{itemize}

\subsection{Monday 4th}
Incorporated Bootstrap Less CSS sources directly into project repository so that
I can hack deeply on the frontend framework rather than just monkey-patch it.

\subsection{Tuesday 5th}
Completed basic sketches for all page mockups. Next task is to tidy up the
stylesheets and incorporate it better into the Boostrap sources.

\paragraph{Le grande plan for great success}
\begin{enumerate}
\item Configure Bootstrap theme
\item Port D1 mockups to Boostrap components
\item Add MySQL user accounts backend
\begin{itemize}
\item Deliverable: setup.php which creates MySQL tables
\end{itemize}
\item Add PHP user accounts controller
\begin{itemize}
\item Deliverable: API with unit tests
\end{itemize}
\item Link frontend with controller for user accounts
\begin{itemize}
\item Deliverable: Functional login page and ubar
\end{itemize}
\item Add MySQL payload backend
\begin{itemize}
\item Deliverable: setup.php which creates MySQL tables
\end{itemize}
\item Add PHP payload controller
\begin{itemize}
\item Deliverable: API with unit tests
\end{itemize}
\item Link frontend with controller for payload
\begin{itemize}
\item Deliverable: Functional details pages
\end{itemize}
\item Develop search controller
\begin{itemize}
\item Deliverable: Functional advanced search page
\end{itemize}
\item Further search controller development
\begin{itemize}
\item Deliverable: inline search component
\end{itemize}
\end{enumerate}

\subsection{Wednesday 6th}
Added arguments to configure in the form \texttt{--enable-feature} which can be
used for enabling minifying, content hashing, local exporting, etc. Used Cogl's
configure.ac as a template for the refactors.

\paragraph{TODO}
\begin{itemize}
\item Research formal HCI methods for web design. Screen recording software for
  user testing?
\end{itemize}

\paragraph{Notes for next meeting with Ian}
\begin{itemize}
\item Build system - le grande demonstration. Will write blog post detailing how
  it works/how to use it.
\item Frontend - Bootstrap, using Less CSS sources.
\item D1 mockups - static prototypes.
\item Next progress - the ``grande plan for great success''.
\end{itemize}

\subsection{Thursday 7th}
Incorporated the site style into Bootstrap, and defined the first two PHP
functions:

\begin{verbatim}
function get_header( $inline_search = false, $value = null,
                     $login_only = false );

function get_footer();
\end{verbatim}

\subsection{Friday 8th}
\paragraph{Notes from meeting with Ian} The theme of this week's meeting is: PROCESS, PROCESS, PROCESS.
\begin{itemize}
\item I need a more definite set of goals for this construction phase.
\item Bring a copy of the Gantt chart to each meeting.
\item Make more constructive notes on the research and planning work done.
\end{itemize}

\paragraph{Notes for next meeting with Ian}
\begin{itemize}
\item HCI - The next round of design work is at the start of next term. The
  remainder of effort this term will be focused on implementation, although I
  will still be checking design with Darren in the meantime.
\item Software architecture planning - Discuss analysis of WordPress, previous
  FYP project, dataset analysis, and database designs.
\item Development plan - Le grande 8 step plan.
\item Development process - show the GitHub issue tracker and milestones. Issues
  are created on a ``as needed'' basis, hence the second term milestones have
  nothing. Show \texttt{./tools/worflow} and development flow diagram.
\end{itemize}

\paragraph{TODO}
\begin{itemize}
\item Complete issues for `Draft Project Plan submission'.
\item Make a write-up of analysis of WordPress source code, previous FYP
  sources, and any other similar projects.
\item Further research of EBI.
\item Research PHP templating.
\end{itemize}

\paragraph{Notes on WordPress} found a detailed description of the
WordPress database design \cite{WordPressND}. There is also detailed
documentation for all of the APIs \cite{WordPressNDa}.
\begin{itemize}
\item There are filesystem and database APIs to completely abstract the
  underlying system. The user should never have to write system or MySQL calls.
\item There is a global \$wpdb variable which is used to talk to the WordPress
  database. API:
\begin{verbatim}
    $wpdb->delete( $table, $where, $where_format = null );
    $wpdb->get_col( 'query', column_offset );
    $wpdb->get_results( 'query', output_type );
    $wpdb->get_row('query', output_type, row_offset);
    $wpdb->get_var( 'query', column_offset, row_offset );
    $wpdb->insert( $table, $data, $format );
    $wpdb->query('query');
    $wpdb->replace( $table, $data, $format );
    $wpdb->update( $table, $data, $where, $format = null, $where_format = null );
    $wpdb->show_errors();
    $wpdb->hide_errors();
    $wpdb->get_col_info('type', offset);
    $wpdb->flush();
\end{verbatim}
\item ID fields in each table are prefixed with the table name and are of type
  BIGINT(20).
\item A \texttt{wp\_options} table stores the settings for the site.
\item A file \texttt{wp-admin/includes/schema.php} contains the database
  information and description.
\end{itemize}

\paragraph{PHP best practises} Read up on how PHP is not object orientated
\cite{Kimsal2011}, and PHP templating practises \cite{Rakowski2011}.

\subsection{Saturday 9th}
\paragraph{Notes on last FYP} 2533 lines of PHP. File structure:
\begin{verbatim}
-rw-r--r-- 1 chris chris  465 Dec 21  2011 ajax.js
-rw-r--r-- 1 chris chris 1.9K Mar 22  2012 ajax.php
-rw-r--r-- 1 chris chris  311 Mar 23  2012 back-up.php
-rw-r--r-- 1 chris chris 3.6K Apr  3  2012 backup.php
-rw-r--r-- 1 chris chris    0 Mar 23  2012 backup.sql
-rw-r--r-- 1 chris chris  238 Apr 20  2012 bottom.php
-rw-r--r-- 1 chris chris  110 Sep 10  2012 db-details.php
-rw-r--r-- 1 chris chris  15K Apr 25  2012 display.php
-rw-r--r-- 1 chris chris 2.8K Apr 20  2012 download.php
-rw-r--r-- 1 chris chris 1.1K Feb  6  2012 funcs.php
drwxr-xr-x 2 chris chris 4.0K Apr  3  2012 images
-rw-r--r-- 1 chris chris 5.3K Mar 22  2012 index.php
drwxr-xr-x 2 chris chris 4.0K Apr 25  2012 js
-rw-r--r-- 1 chris chris 7.7K Apr 16  2012 login.php
-rw-r--r-- 1 chris chris 1.8K Feb 15  2012 protein.php
-rw-r--r-- 1 chris chris 3.1K Mar 14  2012 record.php
-rw-r--r-- 1 chris chris 7.7K Apr 16  2012 restore.php
-rw-r--r-- 1 chris chris  24K Apr 30  2012 search.php
-rw-r--r-- 1 chris chris 4.4K Mar 22  2012 style.css
-rw-r--r-- 1 chris chris 3.3K Apr 20  2012 top.php
-rw-r--r-- 1 chris chris  22K Apr 20  2012 upload.php
drwxr-xr-x 2 chris chris 4.0K Sep 10  2012 uploads

./images:
total 620K
-rw-r--r-- 1 chris chris 1.4K Apr  3  2012 homeico.png
-rw-r--r-- 1 chris chris 7.4K Apr  3  2012 home.png
-rw-r--r-- 1 chris chris  88K Feb 10  2012 logo.png
-rw-r--r-- 1 chris chris 486K Feb  8  2012 logo.psd
-rw-r--r-- 1 chris chris  31K Jan 16  2012 randprot.png

./js:
total 108K
-rw-r--r-- 1 chris chris  92K Jan 17  2012 jquery-1.7.1.min.js
-rw-r--r-- 1 chris chris 5.3K Apr 25  2012 jquery.js
-rw-r--r-- 1 chris chris 5.3K Apr 25  2012 jquery.php

./uploads:
total 2.5M
-rw-r--r-- 1 chris chris 553K Mar 13  2012 database_egle_bunkute2.csv
-rw-r--r-- 1 chris chris 1.1M Mar 13  2012 pidb_1981.csv
-rw-r--r-- 1 chris chris  26K Mar 13  2012 pidb_1995.csv
-rw-r--r-- 1 chris chris 903K Mar 13  2012 pidb_brenda.csv
\end{verbatim}

\paragraph{top.php}
\begin{enumerate}
\item Sets the session
\item Connects to mysql
\item Sets a \texttt{\$root\_dir} variable
\item Prints HTML header
\end{enumerate}

\paragraph{funcs.php} Defines \texttt{get\_record(\$record\_if)}.

\paragraph{bottom.php} Prints out a page footer and closes MySQL connection.

\paragraph{index.php}
\begin{enumerate}
\item Include \texttt{top.php}
\item Output page contents with inline PHP logic
\item Connects to mysql
\end{enumerate}

\paragraph{search.php} 515 lines of intermingled MySQL commands, PHP logic and
HTML.

\paragraph{General notes}
\begin{itemize}
\item Multiple copies of constants scattered throughout sources
\item Spaghetti code source files - a mixture of inline PHP and HTML.
\item MySQL commands embedded into page sources - no MVC separation.
\end{itemize}

\subsection{Sunday 10th}
\paragraph{Website design} found a good checklist of annoying web design
decisions to avoid \cite{Saltman2013}.

\paragraph{Unit testing} read about Zombie.js, a Node.js framework for insanely
fast browser testing (looks awesome!) \cite{Loire2012}.

\subsection{Monday 11th}
Found out about SheetJS - a pure-JavaScript excel parser
\cite{SheetJS2013}. This could be useful for client-side parsing of uploaded
data.

\subsection{Tuesday 12th}
Setup Twig for server-side rendering of HTML templates.

\subsection{Wednesday 13th}
Imported Bootswatch Flatly theme.

\paragraph{TODO}
\begin{itemize}
\item Fixup color scheme for flatly
\item Reduce vertical padding for items
\end{itemize}

\subsection{Thursday 14th}
\paragraph{Notes from meeting with Ian}
\begin{itemize}
\item TODO: Turn the ``le grande 8 steps'' into a set of prototype requirements
  (functional and non-functional), with each requirement being testable and
  QUANTIFIABLE with success criteria.
\item TODO: Create an elaboration and construction plan with requirements
  (functional and non-functional), with each requirement being testable and
  QUANTIFIABLE with success criteria.
\item When making decisions (design or implementation), use the smallest
  experiment which can isolate the decision.
\item TODO: Organise next meeting with Darren to discuss EBI technologies,
  usability design and database design.
\item TODO: So side-by-side implementations of object orientated vs. non-object
  orientated library function.
\item TODO: Create a set of functional and non functional requirements for the
  project based on objectives.
\item Design decisions must be quantifiable and testable, not vague like
  ``should be user friendly''.
\item TODO: For next meeting, demo google analytics on Myrmidon books website.
\item TODO: Add a requirements analysis to project plan.
\end{itemize}

\subsection{Friday 15th}
\paragraph{TODO}
\begin{itemize}
\item Read up on phpDocumentor.
\end{itemize}

\subsection{Saturday 16th}
Read up on MVC architecture in PHP \cite{Butler2010} and some decent arguments
again static variables and methods \cite{Butler2012}. Google has a nice guide to
writing testable code and code smells and symptoms of bad design to look for
\cite{Hevery2008}.

\subsection{Sunday 17th}
Began implementing MySQL backend in library file \texttt{db.php}.

\subsection{Tuesday 19th}
Implement a \texttt{./tools/mkrelease} script which can be used to bump version
number and create a release branch/tag. Bumped project version to 0.0.2.

\subsection{Wednesday 20th}
Have been reading about Dependency Injection in PHP, have found a good intro
article \cite{Rodrigue2012} which recommends a tool Symphony
\cite{SensioLabsND}.

\subsection{Thursday 21st}
\paragraph{Notes from meeting with Ian}
\begin{itemize}
\item Requirements should be uniquely identified, for example, the milestone
  requirements could be identified with M1.1, D3.2, etc.
\item Requirements should be incredibly DETAILED. They are contractual, so the
  ability to irrefutably determine whether a requirement has been met or not is
  essential.
\item Requirements for each iteration should be created at the start, before
  beginning work. The start of next term will begin with generating a set of
  requirements for the second iteration.
\item TODO: Write a set of testable requirements for this first
  elaboration/construction iteration.
\item TODO: Continue/complete development of prototype.
\end{itemize}

\subsection{Friday 22nd}
\paragraph{Notes from meeting with Darren and Fraser}
\begin{itemize}
\item Prototype looks very good, and is mostly what would be expected of the
  final design (save for the unnecessarily spartan homepage).
\item Advanced Search is a high priority TODO, as this will enable user
  testing.
\item Fraser suggested posting to the Life and Health Sciences department a
  request for user testers, or a survey which suggests whether people find it
  useful, or how they would use such a tool.
\item The Enzyme Commission number is a property of enzymes, and not all
  proteins listed in the dataset are enzymes, so not all will have EC numbers.
\item The ONLY two prerequisites for data being entered into the repository are
  that it should contain a protein and an isoelectric point.
\item Fraser noticed that some records which had Greek letters within their
  names (like beta) were displayed incorrectly.
\item I should look through the number of unique values within the dataset to
  determine whether the advanced search page should have a dropdown selector or
  a free-text field.
\item Enzyme Commission number is a hierarchical search property, so people may
  want to start searching with the top (leftmost) value, then specify additional
  tiers. E.g. 3..., 3.1.., 3.1.2., 3.1.2.4. It's unlikely you'd ever want to
  search for ..2.0.
\item It's hard to anticipate how users would want to query the database, so it
  would be nice to keep a record of all the queries that people make, and to use
  this to influence the design. Additionally, we could record IP address of the
  user so as to help collate the number of different users.
\item The purpose of the database is to act as a passive repository of
  information that has been gathered. It is a carbon copy of a set of research
  data, so shouldn't make any assumptions or assertions about the data,
  i.e. don't process the data. To this end, my super-overly-optimised database
  design is crap, and we'd be much better off with ``documents'', not ``relational
  data''.
\item One possibility for administering the dataset could be to allow anyone to
  create an account and upload data, but for this data to be help in a ``holding
  bay'', pending approval from an admin. The admin would then see that someone
  has submitted new data and could review/accept it.
\item The cost of domain registration should come from my project budget, but
  could be done in the Uni's name. Should sort this out in next meeting with
  Ian.
\item Example use cases:
  \begin{enumerate}
    \item A user is writing a paper on a certain protein, and would like to
      quickly look up its isoelectric point. In this case, they would search for
      the protein by name and location (species).
    \item A user would like to see all records for a given EC value (for
      example, 3.1).
  \end{enumerate}
\end{itemize}

\subsection{Saturday 23rd}
Have been given an account and credentials on one of the PSO uni server by Kate
Samperi. URL: \texttt{http://pso.aston.ac.uk/~cummince}.

\subsection{Monday 25th}
Deployed the prototype website to the Aston server (note to self: I should look
into one of the task-runners like Grunt for generating deployment build, the
number of configure arguments is getting pretty unwieldy!). Notes on initial
feedback from Darren:

\begin{itemize}
\item Generally looking positive. The search features that aren't yet
  implemented make it quite hard to test (advanced search, empty fields etc.).
\item Homepage is too sparse (address this in D2).
\item The dropdown for experimental methods show lots of duplicates, i.e. the
  data isn't standardised.  Perhaps it would be possible to keep two separate
  lists of methods - one which contains the methods that were entered ``as is''
  (with all the duplicates), and a separate, standardised list which can be used
  for searching (but would be hidden from the user). I'll take a look into this
  - it could provide the right balance between keeping an honest record of the
  data ``as recorded'', and constructing a standardised and relational model for
  categorising the data.
\end{itemize}
