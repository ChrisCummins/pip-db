\section{October 2013 \hrulefill}

\subsection{Monday 7th}
I've organised a meeting with Ian Nabney (i.t.nabney@aston.ac.uk) for Thursday
10th 10:30am in MB216A. Topics for discussion:

\begin{itemize}
\item Dataset – getting a better understanding of the data which I will be
  working with, the scientific background behind it, how often it will be
  updated, whether it is confidential/open source etc.
\item Users – what is the expected userbase, what are their expectations,
  concerns, and technical experience? Etc.
\item Time schedule – should I be using any development processes? How often
  should I be meeting/reporting back? Etc.
\item Difference(s) between the protein diffusion coefficient projects.
\item Additional project requirements – what is the expected site traffic? High
  or low? Do I need to organise dedicated servers?
\item Any existing related databases to use as reference.
\item Any confidentiality/plagiarism issues with hosting WIP code on github
  (effectively open sourcing the whole project)?
\end{itemize}

\subsection{Wednesday 9th}
Some personal goals for the project:

\begin{enumerate}
\item To produce something of value to the molecular biology community.
\item To exclusively use open source (free as in freedom and free as in free
  beer) software and tools, without exception.
\item To learn more
\item To get a high mark ($> 70\%$) mark for all assessments.
\item To contribute back to at least one of the tools that I use in some
  quantifiable way.
\end{enumerate}

\subsection{Thursday 10th}
Met with Ian. Notes:

\begin{itemize}
\item Choice of process is important - look up RUP (IBM Rational Unified
  Process) \cite{IBMRUP} and consider using it. If not, suggest an improved
  alternative and justify.
\item Choice of language is personal preference. Ian has experience with PHP
  making it a solid choice – but a justified alternative is acceptable.
\item I should meet with Dr. Darren Flower (Aston lecturer, effectively product
  owner for this project) ASAP to discuss the project, and he can give me the
  data and help answer questions about it.
\item I can use \texttt{github.com} to host code, but shouldn't upload the bio
  data itself.
\item First term work is for a lot more than just planning stage. I should be
  using it to begin work properly and use it to brush up on MySQL/PHP skills as
  required.
\item Weekly meetings are essential, 10:30 on Thursdays.
\end{itemize}

\paragraph{TODO}
\begin{itemize}
\item Read up on RUP.
\item Contact Darren to arrange meeting.
\item Read about git and MySQL \cite{Kulbertis2011}.
\end{itemize}

\paragraph{Unified Process} The Unified Software Development Process is an
iterative and incremental software development process framework. The basis of
the process is that it is a use case driven iterative and incremental process,
in which Elaboration, Construction and Transition are divided into a series of
time boxed phases and iterated upon.

\paragraph{IBM Rational Unified Process} An iterative software development
process framework created by IBM (Rational Software Corporation division), which
is an implementation of the Unified Process. It details six best practises for
modern software engineering:

\begin{enumerate}
\item Develop iteratively, with risk as the primary iteration driver
\item Manage requirements
\item Employ a component-based architecture
\item Model software visually
\item Continuously verify quality
\item Control changes
\end{enumerate}

\noindent
Each iteration contains nine disciplines. Six engineering disciplines:
\begin{enumerate}
\item Business modelling
\item Requirements
\item Analysis and Design
\item Implementation
\item Test
\item Deployment
\end{enumerate}

\noindent
And three supporting disciplines:
\begin{enumerate}
\item Configuration and change management
\item Project management
\item Environment
\end{enumerate}

\noindent
The RUP project life cycle consists of four phases:
\begin{enumerate}
\item Inception phase
\item Elaboration phase
\item Construction phase
\item Transition phase
\end{enumerate}

\paragraph{Problems with RUP} It is isn't open source. A user license for the
Rational Method Composer must be bought from IBM for \$1,080.00 dollars, and so
I must look for alternatively so as not to violate my personal goal of using
purely open source tools/frameworks.

\paragraph{OpenUP} The Open Unified Process, alternatively known as the Eclipse
Process Framework (EPF). This offers an open source minimalist implementation of
RUP.

\begin{quote}
  OpenUP is a lean Unified Process that applies iterative and incremental
  approaches within a structured life-cycle. OpenUP embraces a pragmatic, agile
  philosophy that focuses on the collaborative nature of software
  development. It is a tools-agnostic, low-ceremony process that can be extended
  to address a broad variety of project types.
\end{quote}

\noindent
Arranged meeting with Darren for 2pm in MB449 tomorrow.

\subsection{Friday 11th}
Purposes of the project planning module:

\begin{enumerate}
\item Develop a project plan.
\item Produce a schedule of activities using GANTT charts or similar.
\item Consider which activities are critical to the success of your project.
\item Undertake a risk analysis.
\item Ensure that tangible milestones are identified to measure progress and
  success.
\item Determine the resources needed to carry out and complete your project
  work.
\end{enumerate}

\noindent
From the guidance notes, a good project topic:

\begin{enumerate}
\item Provides a sufficient intellectual challenge. – will be determined by 3).
\item Contains a relatively straightforward core part which should be achievable
  even in the worst-case scenario. – Yes. The creation of an online database.
\item Has an element of innovation, not necessarily in the research sense, but
  at least in the sense of ‘not covered or detailed in final-year lectures or in
  standard textbooks'. – Not sure about this.
\item Yields results that can be verified in some way. – Yes.
\end{enumerate}

\noindent
Questions for next meeting with Ian:
\begin{itemize}
\item Availability of dedicated servers, and public IPs.
\item Registration of a domain.
\item Clarify the purpose of the project logbook: is it for my benefit or the
  assessor's?
\item Confirm access to the database module notes.
\end{itemize}

\noindent
Questions for meeting with Darren:

\begin{itemize}
\item The Dataset:
  \begin{itemize}
  \item What is it used for?
  \item Who is it used by?
    \begin{itemize}
    \item Confidential or open source?
    \item Publicly available or user account based access?
    \end{itemize}
  \item How often is it used? (Site traffic)
  \item How often is it updated?
    \begin{itemize}
    \item User submissions or single admin account?
    \end{itemize}
  \item Feedback from users?
    \begin{itemize}
    \item Reddit style popularity mechanism?
    \item Comments system?
    \end{itemize}
  \end{itemize}
\item The website:
  \begin{itemize}
  \item Possible volunteers for user testing?
  \item Associated with Aston?
  \end{itemize}
\end{itemize}

\noindent
Notes from meeting with Darren:
\begin{itemize}
\item There are hundreds (/thousands) of existing databases which offer specific
  categories of information which tend to interlink with each other.
\item Many different use cases for looking up data. Some people will want a dump
  of the whole dataset so as to perform calculations, some other cases may be
  interested in performing more targeted searches, such as by type/range of pHs
  etc.
\item A BLAST search implementation would be useful.
\item Possible visualisations of results such as graphing distribution of pI
  ranges would be useful.
\item Fuzzy matching of search criteria would be useful.
\item All data contains citations to original source.
\item User submitted data could be potentially useful, but would need a way to
  distinguish it from the `authoritative' source. I think this could be a really
  interesting topic of research – how can you authenticate user submitted data?
  Ranking by the weight of the citation? User rating? Peer reviewing?
\item Version control of the dataset would be useful. For example, viewing
  history (like Wikipedia's page history), and the ability to undo revisions.
\item Site traffic is impossible to predict. It could be used by lots of people,
  or it may not. To accommodate this, I should assume worst-case scenarios for
  the number of users – in cases of peer reviewing, assume low userbase; in
  cases of site traffic, assume heavy load.
\item Darren has volunteered for offering user testing of work-in-progress
  sites.
\item Darren would like to hear from me with progress reports, or any questions
  etc. Should keep him in the loop. Perhaps periodic progress reports sent by
  email?
\item Fraser is a final year Biology student who has worked with the dataset
  extensively, and should be able to help go through the data with me and
  explain what it means.
\item Darren will send me the dataset in the near future.
\end{itemize}

\noindent
Site security concerns:

\paragraph{Confidentiality} Any user system will require user-names and passwords
to remain confidential.
\paragraph{Integrity} Integrity of dataset is CRUCIAL. Any unwarranted
modifications or loss of data would invalidate the whole project.
\paragraph{Accessibility} Data must be accessible at all times. System downtime
is unacceptable, and this means that updates to dataset must occur transparently
to the user. This also includes making offline copies of the data for users.

\paragraph{Project Goal} A reasonable personal goal would be to get the basic
search-able database online by the end of the year. This provides ample time for
adding the ``fun'' stuff, such as visualisation, advanced searching techniques,
extensive user testing, site design etc. etc.

Have received dataset from Darren. Note that it is currently CONFIDENTIAL ``in
so far as you shouldn't pass it on to anyone else, or otherwise make it
available to others''.

\paragraph{Development Plan}
\begin{enumerate}
\item Setup MySQL database and suitable tables.
\item Create a web page to add entries to db and test.
\item Create a script to import exiting data in spreadsheets.
\item Create forms to search database.
\item Create forms to download database/search queries.
\end{enumerate}

\paragraph{UI Idea} use an accordion model to hide further details of search
requests and show them on demand, instead of showing a list of links to separate
details pages.

\paragraph{UI Idea} use a minimalist front page if we don't have anything to
show. Think \texttt{google.com}.

\subsection{Sunday 13th}
Found that the PHP sources for a couple of pages from Facebook were leaked in
2007, now available on github \cite{Buvrilovic2013}. Should keep tabs on this so
as to see how a large database backed website organises its code.

\noindent
Notes on Facebook source code:
\begin{itemize}
\item Function \texttt{get\_site\_variable()} provides API into state vars.
\item \texttt{IS\_DEV\_SITE} and other globals provide some debugging.
\item \texttt{xxx\_stats} variable arrays are used for grouping relevant data.
\item Directory structure:
  \begin{itemize}
  \item \texttt{/html/}
  \item \texttt{/libs/}
  \item \texttt{/js/}
  \item \texttt{/css/}
  \end{itemize}
\item Templates are used and rendered last:
\begin{verbatim}
    render_template($_SERVER['PHP_ROOT'].'/html/index.phpt');
\end{verbatim}
\end{itemize}

\subsection{Monday 14th}
Read up on PHP templating \cite{Rakowski2011}, should investigate use of Twig
framework to implement MVC architecture in my project.\\

\noindent
Amended meeting time with Fraser to 4pm on Wednesday.\\

\noindent
Started generating project plan (Gantt chart). OpenUP reference:
\texttt{http://epf.eclipse.org/wikis/openup/}.

\subsection{Tuesday 15th}
Created first iteration of site mockups to get a feel for the
design. Requirements are still changing rapidly so this helps formulate a TODO
list.

\paragraph{Back-end idea} Don't just store external URLs for references, instead
break it down logically into a website name, author, unique ID, etc.

\paragraph{Plan for site map}
\begin{itemize}
\item Home
  \begin{itemize}
  \item Help
  \item About Us
  \item Terms \& Conditions
  \item Privacy Policy
  \item Contact Us
  \end{itemize}
\item Search Results
  \begin{itemize}
  \item Download Results
  \item Individual Record
  \item Edit Record
  \item Add New Record
  \end{itemize}
\item Members Page
  \begin{itemize}
  \item Register
  \item Login
  \end{itemize}
\end{itemize}

\noindent
Had meeting first with Fraser. Notes:

\begin{itemize}
\item Dataset consist of multiple sheets which can be combined, with an extra
  field to note the sheet it originated from (1975, PubMed, etc.). Note that
  data in sixth form sheet may be incorrect or less valid.
\item Notes, by field:
  \begin{itemize}
  \item \textbf{E.C.} Enzyme Commission Number. Numerical classification scheme,
    consisting of four positive integer values. Many to one relationship of
    proteins and records to E.C. value. Can be unknown.
  \item \textbf{Protein / Alternative Name(s)} string names of equal value.
  \item \textbf{Source} Latin binomial. Case sensitivity is important:
    capitalise the Genus, everything else is lower case. Does have common names
    (property).
  \item \textbf{Organ and/or Subcellular location} a property of the source.
  \item \textbf{M.W.} Molecular Weight, in units of Daltons (Da).
  \item \textbf{Subunit No. / M.W (range)} properties of molecular weight.
  \item \textbf{No. of Iso-enzymes} Values can be vague (e.g. Many/several), not
    that important.
  \item \textbf{pI maximum value / pI range / pI value of major component / pI}
    various ways to specify pI value with varying precision.
  \item \textbf{Temperature} Important for replicating experimental results.
  \item \textbf{Method} Experimental method
  \item \textbf{Valid sequence(s) available} Values in the fields are shorthand,
    use the key at the bottom of each sheet to decode. Entries with `1' are more
    robust and should be ranked greater than entries with `0' values.
  \item \textbf{Protein sequence / Species Taxonomy / Original Texts / PubMed}
    External links.
  \item \textbf{Notes} text with additional details. Could be combed through to
    see if info could be added to other fields.
  \end{itemize}
\end{itemize}

\paragraph{Prototyping idea} Write a ``Plausible Nonsense Generator'' which can
create fake but believable ideal datasets for testing with.

\subsection{Wednesday 16th}
Notes from weekly meeting with Ian:

\begin{itemize}
\item No news yet on availability of server/IP.
\item Registration of domains can be done whenever (he recommended quiet late,
  I'll recommend quite early), small cost will be reimbursed by department.
\item Refer to Kate for answers on: logbook style (is it for my benefit or
  assessors'?), the distribution of marks between testing and evaluation, term
  dates.
\item `Balsamiq' is a UI prototyping tool which is useful for generating
  wireframes and mockups. I should look into this.
\item First iteration of Gantt chart/project plan is lacking in User
  requirements and risk assessment time. More time should be dedicated to this,
  don't rush into implementation. Also need to propose a business case (although
  this will be pretty minimal: more science = better humanity).
\item A proper audit of the previous student's project should be done and
  recorded (this can be added to project plan). This can be chalked up as
  ``Contextual Investigation''
\item Be more honest in risk assessment. Include things like ``need to learn
  PHP'', ``I'm a MySQL n00b'', etc.
\end{itemize}

\paragraph{TODO}
\begin{itemize}
\item Second iteration of Gantt chart/project plan.
\item Write an assessment of project risks.
\item Talk with Darren about further user investigation.
\item Audit the previous FYP.
\item Download and test Balsamiq.
\end{itemize}

\subsection{Thursday 17th}
\paragraph{An examination of pidb back-end}
\begin{verbatim}
mysql> SHOW TABLES;
+------------------+
| Tables_in_pidb   |
+------------------+
| Locations        |
| MethodCollection |
| Methods          |
| ProteinAltNames  |
| Proteins         |
| Records          |
| Settings         |
| Sources          |
| TextCollection   |
| Texts            |
| Users            |
+------------------+
11 rows in set (0.00 sec)

mysql> DESCRIBE Locations;
+------------+--------------+------+-----+---------+----------------+
| Field      | Type         | Null | Key | Default | Extra          |
+------------+--------------+------+-----+---------+----------------+
| LocationID | int(15)      | NO   | PRI | NULL    | auto_increment |
| Location   | varchar(255) | NO   |     | NULL    |                |
+------------+--------------+------+-----+---------+----------------+
2 rows in set (0.00 sec)

mysql> DESCRIBE MethodCollection;
+----------+---------+------+-----+---------+-------+
| Field    | Type    | Null | Key | Default | Extra |
+----------+---------+------+-----+---------+-------+
| RecordID | int(15) | NO   | PRI | NULL    |       |
| MethodID | int(15) | NO   | PRI | NULL    |       |
+----------+---------+------+-----+---------+-------+
2 rows in set (0.02 sec)

mysql> DESCRIBE Methods;
+----------+--------------+------+-----+---------+----------------+
| Field    | Type         | Null | Key | Default | Extra          |
+----------+--------------+------+-----+---------+----------------+
| MethodID | int(15)      | NO   | PRI | NULL    | auto_increment |
| Method   | varchar(255) | NO   |     | NULL    |                |
+----------+--------------+------+-----+---------+----------------+
2 rows in set (0.00 sec)

mysql> DESCRIBE ProteinAltNames;
+-----------+--------------+------+-----+---------+-------+
| Field     | Type         | Null | Key | Default | Extra |
+-----------+--------------+------+-----+---------+-------+
| ProteinID | int(15)      | NO   | PRI | NULL    |       |
| AltName   | varchar(255) | NO   | PRI | NULL    |       |
+-----------+--------------+------+-----+---------+-------+
2 rows in set (0.02 sec)

mysql> DESCRIBE Proteins;
+--------------+--------------+------+-----+---------+----------------+
| Field        | Type         | Null | Key | Default | Extra          |
+--------------+--------------+------+-----+---------+----------------+
| ProteinID    | int(15)      | NO   | PRI | NULL    | auto_increment |
| Name         | varchar(255) | NO   |     | NULL    |                |
| SequenceLink | text         | NO   |     | NULL    |                |
+--------------+--------------+------+-----+---------+----------------+
3 rows in set (0.00 sec)

mysql> DESCRIBE Records;
+----------------+--------------+------+-----+---------+----------------+
| Field          | Type         | Null | Key | Default | Extra          |
+----------------+--------------+------+-----+---------+----------------+
| RecordID       | int(15)      | NO   | PRI | NULL    | auto_increment |
| EC             | varchar(255) | NO   |     | NULL    |                |
| ProteinID      | int(15)      | NO   |     | NULL    |                |
| SourceID       | int(15)      | NO   |     | NULL    |                |
| LocationID     | int(15)      | NO   |     | NULL    |                |
| MWMin          | int(50)      | NO   |     | NULL    |                |
| MWMax          | int(50)      | NO   |     | NULL    |                |
| SubUnitNo      | int(5)       | NO   |     | NULL    |                |
| SubUnitMW      | varchar(255) | NO   |     | NULL    |                |
| IsoEnzymesMin  | varchar(255) | NO   |     | NULL    |                |
| IsoEnzymesMax  | varchar(255) | NO   |     | NULL    |                |
| PIMaxValue     | float        | NO   |     | NULL    |                |
| PIMin          | float        | NO   |     | NULL    |                |
| PIMax          | float        | NO   |     | NULL    |                |
| PIMajorComp    | float        | NO   |     | NULL    |                |
| PIValue        | float        | NO   |     | NULL    |                |
| TemperatureMin | int(255)     | NO   |     | NULL    |                |
| TemperatureMax | int(255)     | NO   |     | NULL    |                |
| PubMed         | text         | NO   |     | NULL    |                |
| Notes          | varchar(255) | NO   |     | NULL    |                |
+----------------+--------------+------+-----+---------+----------------+
20 rows in set (0.01 sec)

mysql> DESCRIBE Settings;
+-------+--------------+------+-----+---------+----------------+
| Field | Type         | Null | Key | Default | Extra          |
+-------+--------------+------+-----+---------+----------------+
| ID    | int(15)      | NO   | PRI | NULL    | auto_increment |
| Name  | varchar(255) | NO   |     | NULL    |                |
| Value | varchar(255) | NO   |     | NULL    |                |
+-------+--------------+------+-----+---------+----------------+
3 rows in set (0.00 sec)

mysql> DESCRIBE Sources;
+--------------+--------------+------+-----+---------+----------------+
| Field        | Type         | Null | Key | Default | Extra          |
+--------------+--------------+------+-----+---------+----------------+
| SourceID     | int(15)      | NO   | PRI | NULL    | auto_increment |
| Name         | varchar(255) | NO   |     | NULL    |                |
| TaxonomyLink | text         | NO   |     | NULL    |                |
+--------------+--------------+------+-----+---------+----------------+
3 rows in set (0.01 sec)

mysql> DESCRIBE TextCollection;
+----------+---------+------+-----+---------+-------+
| Field    | Type    | Null | Key | Default | Extra |
+----------+---------+------+-----+---------+-------+
| RecordID | int(15) | NO   | PRI | NULL    |       |
| TextID   | int(15) | NO   | PRI | NULL    |       |
+----------+---------+------+-----+---------+-------+
2 rows in set (0.00 sec)

mysql> DESCRIBE Texts;
+----------+--------------+------+-----+---------+----------------+
| Field    | Type         | Null | Key | Default | Extra          |
+----------+--------------+------+-----+---------+----------------+
| TextID   | int(15)      | NO   | PRI | NULL    | auto_increment |
| Link     | text         | NO   |     | NULL    |                |
| TextType | varchar(255) | NO   |     | NULL    |                |
+----------+--------------+------+-----+---------+----------------+
3 rows in set (0.00 sec)

mysql> DESCRIBE Users;
+----------+--------------+------+-----+---------+----------------+
| Field    | Type         | Null | Key | Default | Extra          |
+----------+--------------+------+-----+---------+----------------+
| UserID   | int(15)      | NO   | PRI | NULL    | auto_increment |
| Name     | varchar(255) | NO   |     | NULL    |                |
| Email    | varchar(255) | NO   |     | NULL    |                |
| Password | varchar(32)  | NO   |     | NULL    |                |
| Type     | varchar(50)  | NO   |     | NULL    |                |
+----------+--------------+------+-----+---------+----------------+
5 rows in set (0.00 sec)
\end{verbatim}

\subsection{Friday 18th}
\paragraph{Balsamiq} Downloaded and tested Balsamiq (trial addition). It's
absolutely perfect for my needs, should but User license when trial
expires. Should add a `plan' directory to github repo for storing mockups.

\paragraph{TODO}
\begin{itemize}
\item Create mockups of previous FYP site designs.
\item Create mockups of current site designs.
\end{itemize}

\paragraph{Database design} Have created some first ideas for database designs
using UML database notation. Should add PDF exports of these to repo.

\paragraph{Data repetition within dataset} Have been auditing the dataset that
Darren sent me. There's an awful lot of repetition of data in almost every
field, so the database design should be heavily normalised to optimise for this.

\subsection{Saturday 19th}

\paragraph{TODO}
\begin{itemize}
\item Perform full audit of dataset
\item Develop Plausible Nonsense Generator
\end{itemize}

\paragraph{Plausible Nonsense Generator} Started developing PNG as a HTML/JS
tool for creating nonsense payloads by using a randomised fake dataset. Should
postpone further development on this pending completion of the real dataset
audit so I know how best to mimic it.

\subsection{Sunday 20th}

\noindent
Things to discuss at next FYP meeting with Ian:
\begin{itemize}
\item Analysis of last FYP - Highlight technical differences (database design,
  MVC architecture) and user experience flaws. What are Ian's opinions on the
  implementation. NOTE: I really don't want anything to do with the past
  project beyond using it as a reference.
\item ``Balsamiq'' - super awesome. Show early prototypes.
\item Risk assessment - Examples of risk assessments (obviously Google returns
  nothing useful).
\item Database design - UML diagram.
\item Project planning - second iteration Gantt chart.
\end{itemize}

\subsection{Monday 21st}
I'm starting to get aggravated with the entire software development process side
to the project. The more that I read about RUP and OpenUp, the less I feel that
either process would contribute anything positive to the project. Additionally,
I have been researching recommendations on software development processes for
solo projects and have found not one source that recommends their use for
individual work, with both the official documentation of development processes
anecdotal evidence suggesting that their main value is in organising teams. The
advice that I have seen for solo projects covers things that I already am doing
or intend to do:

\begin{itemize}
\item Pick a good version control system and use it fastidiously.
\item Write down a list of goals and achievements.
\item Keep a log of your progress and decisions made.
\item Document your code.
\item Use a bug tracker.
\end{itemize}

I should check with Kate that not using a development process will not affect my
marks/assessment before ``making a stand'', but I am confident that being
shouldered into having to adopt a process for this project would at best be a
distraction and at worse would jeopardise my efforts by adding artificial
constraints that get in the way of progress. One exception to this is test
driven development, which I have first-hand experience of using from `emu'
\cite{Cummins2013} and `t4' \cite{Cummins2013a}, and is a process which I intend
to use when implementing some of the data back-end.

\paragraph{TODO (planning)}
\begin{itemize}
\item Begin writing project plan.
\item Write a set of use cases for common tasks, and create mockups showing how
  to perform those tasks with pidb/protein-db.
\end{itemize}

\paragraph{TODO (implementation)}
\begin{itemize}
\item Add probability control to PNG and re-implement as a native application
  (or as a scriptable web app).
\item Read up on Selenium \cite{SeleniumND}.
\item Implement \texttt{make DEBUG=1 all} feature.
\item Move website sources into \texttt{www/} subdirectory.
\end{itemize}

\paragraph{Preparing CSV files from Dataset} Save the desired sheet as text (tab delimiter).

\paragraph{Working with CSV dataset}
\begin{verbatim}
# To remove header line from output
$ cat dataset.csv | tail -n+2

# To print first column
$ cat dataset.csv | awk -F $'\t' '{print $2}'

# To count non-empty fields
$ cat dataset.csv | tail -n+2 | \
  awk -F $'\t' '{print $2}' | sed '/^$/d' | wc -l
\end{verbatim}

\subsection{Tuesday 22nd}
Began creating Objectives, Milestones, Success, Use Cases and Risks tables for
use in project planning meeting tomorrow.\\

\noindent
Created side-by-side mockups of common tasks in pidb and my first iteration
design using Balsamiq.

\subsection{Wedensday 23rd}
\paragraph{Laptop TODO}
\begin{itemize}
\item Install balsamiq trial
\item Install lessc
\item Install tablify
\item Test dsa, png and build system
\end{itemize}

\paragraph{Demos for Ian}
\begin{itemize}
\item Balsamiq mockups
\item Build system
\item dsa
\item png
\end{itemize}

\paragraph{Notes from PP meeting with Kate}
\begin{itemize}
\item Final poster assessment is A1 size - glossy paper looks best.
\item Gantt chart week numbers should include some dates for reference.
\item Gantt chart needs project plan marked on it.
\item Be clear in project plan of which activities depend on each other and
  which milestones are ``blocking'' achievements.
\item There are 26 work weeks in the term.
\item Deadline for preliminary project plan is November 1st.
\item Next PP meeting is a peer review of project plans on November 11th.
\end{itemize}

\subsection{Thursday 24th}
\paragraph{Notes from weekly meeting with Ian}
\begin{itemize}
\item Make a third iteration Gantt chart, and send to Ian.
\item Discuss ``purpose of project'' with Darren at next meeting, not just user
  interface.
\item Make a list of mitigation strategies for each of the risks in the Risk
  list.
\end{itemize}

\noindent
Started using the GitHub issue tracker and imported the milestones and current
task list. Added GitHub pages for the repo, should populate that later. Ideas
for public page: a link to latest PDF build of this log, automatic burndown
charts, a dynamic list of top bugs and most recent commits, etc.

\subsection{Friday 25th}
Prepared Balsamiq mockups for meeting with Darren at 2pm today, and installed
previous FYP onto laptop for demonstration.

\paragraph{Notes from meeting with Darren}
\begin{itemize}
\item D1 initial design is OK.
\item The search engine should provide hard limits on the number of results it
  returns (perhaps as a percentage of the dataset size?); this is to prevent
  users from just downloading the whole dataset themselves and eradicating the
  need for the database.
\item In order to achieve this, search criteria should be split into two types:
  primary and secondary. Primary search criteria are those which can only return
  a finite subset of the dataset, whereas secondary criteria may return the
  entire set, and so should be combined with primary criteria. For example, a
  protein name is a primary criterion, whereas a pI range of 0-14 (all possible
  values) is a secondary criterion.
\item Incorporating Blast searching will involve generating a list of sequences
  for every record and then using that database with the blast search
  program. This is a high priority ``bonus'', as it will be a step above the
  previous attempts.
\item The design is a very subjective thing - as long as it is functional, I
  have a lot of creative control over how the user should interact with the
  site.
\item I have relative freedom over the project name, but Darren will be able to
  contribute suggestions if necessary.
\item I should see if Fraser has anything he would like to say about the design.
\end{itemize}

\subsection{Saturday 26th}
Ported PNG to node.js yesterday. Node.js seems fantastic for what I
want (server-side JS), and the combination of that and MongoDB seems
to be very interesting - it's worth reading more into it.

\paragraph{Reading List}
\begin{itemize}
\item Node.js testing framework - mocha \cite{HolowaychukND} and should.js
  \cite{HolowaychukNDa}
\item Blog rolling with mongoDB \cite{Clemson2010}.
\item Real Time Web with Node.js \cite{CoursewareND}
\item Node.js vs PHP \cite{Webapplog2013}
\end{itemize}

\subsection{Sunday 27th}
Began replacing the static Makefile with an autotooled build system. This will
provide greater portability and is a more suitable process for building the
project as it adds the extra configuration stage which can be used to specify
options such as whether to use content hashing, whether to build the TeX
sources, etc.

\paragraph{Build system TODO}
\begin{itemize}
\item configure: Add --enable-(html|js|css)-minification=(yes|no) feature
\item configure: Check for node else fail
\item configure: Check for dsa runtime dependencies else fail
\item configure: Check for Apache/MySQL else warn
\item automake: Add make install command
\item configure: Add --prefix for www_root
\item configure: Add --enable-content-hashing=(yes|no) feature
\item configure: Add summary
\item automake: Add log make open|edit rules
\end{itemize}
