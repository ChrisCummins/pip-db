\thispagestyle{plain}

%%%%%%%%%%%%%%%%%%
%% Inside title %%
%%%%%%%%%%%%%%%%%%

\begin{center}
  \Large
  \textbf{\@title}

  \vspace{0.4cm}
  \large
  \subtitle{}

  \vspace{0.4cm}
  \@author
\end{center}

%%%%%%%%%%%%%%
%% Abstract %%
%%%%%%%%%%%%%%

\centredtitle{Abstract}

The isoelectric point or pI of a protein corresponds to the solution
pH at which the net surface charge is zero. Since the earliest days of
solution biochemistry, the pI has been recorded and reported, and
literature reports of pI abound. The protein isoelectric point
database (PIP-DB) has collected and collated this legacy data to
provide an increasingly comprehensive database for comparison and
benchmarking purposes. As part of a collaboration between Aston
University's Computer Science and Life and Health Sciences
departments, a web application has been developed to warehouse this
database and provide public access to this important information.

A schema and file format (YAPS) has been designed to house protein
isoelectric point datasets, with appropriate tooling to convert
between PIP-DB and the YAPS format. A search engine has been developed
for the dataset, using a domain specific language which allows for
compound queries to be expressed using tree structures in
LISP. Support for protein sequence searching has been implemented
using the NCBI BLAST+ search tools.

A user-centred approach to designing web applications has been
adopted, with a heavy focus on interaction design and usability
testing. The results of usability testing show numerous advantages in
the interface design, such the design of a widget which dynamically
indicates the number of results to be returned by a search engine
query.

A public API for allowing programmatic communication with the search
engine has been designed. The website makes extensive use of mobile
code, implementing a thin presentation layer wrapper around the
API. The use of mobile code in websites is discussed and a comparison
is made with server-side rendering of HTML.

A unique emphasis has been placed on development of infrastructure,
with several new tools being written for reuse in other
projects. Among these is the novel application of Markov text
generators for creating test payloads from confidential datasets, and
a project management program (pipbot) which automates version control
and build configurations.

A parallelised build system has been developed which provides
homogeneous development and deployment configurations. Tests show how
shell-level parallelism can reduce compilation time by a factor of
5. An implementation of checksum based cache busting and on-demand
build systems using the Linux's inotify subsystem is described.


%%%%%%%%%%%%%%%%%%%%%%
%% Acknoweldgements %%
%%%%%%%%%%%%%%%%%%%%%%

\centredtitle{Acknowledgements}

I would like to thank Ian Nabney for the excellent continued
supervision and guidance, without which this project would not have
been possible. Further thanks to Darren Flower for providing the
dataset and for patiently enduring my lowly understanding of the
natural sciences. I would like to acknowledge Kate Sugden and all of
the academic staff at Aston University who I've had the pleasure of
being taught by. Special thanks to Fraser Crofts, Ben Stone, Shahzad
Mumtaz, Mahmood Jasmin, and Dan Clarke for volunteering their time to
help with usability testing.
