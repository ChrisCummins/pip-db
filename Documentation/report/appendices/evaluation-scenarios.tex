\chapter{Usability testing scenarios}\label{app:evaluation-scenarios}

The following four scenarios are to be completed by the participant
during usability testing.

\section*{1. Researching a specific protein}
\begin{enumerate}
\item You are conducting an experiment which requires you to know the
  isoelectric point (pI) of a protein called \textit{Lactoferrin}.
\item You would also like to know the range of isoelectric points
  (lowest and highest) for all proteins obtained from the same
  \textit{source} as \textit{Lactoferrin}.
\end{enumerate}

\section*{2. Performing broad searches}
\begin{enumerate}
\item In this scenario, you would like to research \textit{Kinase}
  proteins. You would like to download a CSV file which contains all
  proteins which match the following criteria:
  \begin{itemize}
  \item They must contain \textit{Kinase} in their names.
  \item They were obtained from a \textit{Human} source.
  \item Their enzyme commission number begins with the three digits
    \textit{2.7.1}.
  \item They were discovered at a temperature greater than or equal to \textit{4\celsius}.
  \end{itemize}
\item Once you have downloaded the CSV file, open it in a spreasheet
  program and identify the one protein with a Molecular Weight of
  \textit{86,000}.
\end{enumerate}

\section*{3. Further broad searches}
\begin{enumerate}
\item You would like to know the number of entries in the database
  which contain the word \textit{Kinase} in their names, and compare
  this to the number of entries which \textit{do not} contain the word
  \textit{Kinase}.
\item From the entries which do not contain the word \textit{Kinase},
  you would like to find the protein with the \textit{lowest}
  isoelectric point.
\item You would like to know the names of the authors of the PubMed
  article for this protein.
\end{enumerate}

\section*{4. Identifying proteins using FASTA sequence}
\begin{enumerate}
\item You have been supplied with the following protein sequence:
\begin{verbatim}
>sp|P02754|LACB_BOVIN Beta-lactoglobulin OS=Bos taurus GN=LGB PE=1 SV=3
MKCLLLALALTCGAQALIVTQTMKGLDIQKVAGTWYSLAMAASDISLLDAQSAPLRVYVE
ELKPTPEGDLEILLQKWENGECAQKKIIAEKTKIPAVFKIDALNENKVLVLDTDYKKYLL
FCMENSAEPEQSLACQCLVRTPEVDDEALEKFDKALKALPMHIRLSFNPTQLEEQCHI
\end{verbatim}

You would like to identify the protein that this sequence came from,
and download a CSV file containing the details of all protiens which
match this sequence with an isoelectric point within the range
\textit{5.1 - 5.2}.
\end{enumerate}

{\centering END OF SCENARIOS.\par}
