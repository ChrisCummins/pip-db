\chapter{Risk mitigation strategies}\label{app:risk-mitigation}

\paragraph{R1 - Design is not intuitive} The key to mitigation of this
risk is in frequent and effective user testing and an understanding of
typical and common use-cases for the product.

\paragraph{R2 - Project involves use of new technical skills} In order
to prevent this risk from having a serious impact on the project, it
will be necessary to begin studying and reading about the technologies
that will be used at a very early stage in the project, long before
the start of the implementation.

\paragraph{R3 - High Level of technical complexity} Avoiding this risk
will involve ensuring that the scope of the project remains
technically feasible, and that the software architecture is abstracted
into small enough units that it is easier to focus on each one
separately, as well as keeping small iterative development cycles and
adequate test coverage to prevent regressions when implementing new
functionality.

\paragraph{R4 - Complex deployment of production website} A website
with independent data and application logic components can result in
an intricate deployment process. This is a common problem in the
development of complex web application, where development and
production environments must be synchronised and differences between
debugging and releases builds must be accounted for. In order to
mitigate this risk, a suite of tools to configure, build and deploy
the website should be developed at an early stage, allowing for fast
deployment of public releases.

\paragraph{R5 - Project milestones not clearly defined} A thoroughly
described and well thought out project plan will help to prevent
scheduling issues and delays in development that would arise from this
risk.

\paragraph{R6 - System requirements not adequately identified} A
comprehensive specification of the finished product before
implementation begins will help to mitigate this risk.

\paragraph{R7 - Change in project requirements during development}
An agile approach towards accommodating for changes in the
requirements should be used so as to keep the time between user
feedback sessions and input from stakeholders low.

\paragraph{R8 - Changes in dataset format during development} It is
not possible to entirely avoid this risk due its nature and the
dependence on third parties, but steps can be taken to prevent any
delays that this would cause, chiefly, a well abstracted data parsing
component which can be switched and modified if necessary to
accommodate for a new dataset format.

\paragraph{R9 - Unable to obtain required resources} Since the
project does not require many resources, it is important to acquire
these as early on in the development process as possible, and
alternative resources should be planned for, such as local test
servers.

\paragraph{R10, R11, R12 - Users not committed to the project, lack
of cooperation from users, and users with negative attitudes toward
the project} The usefulness of the finished project will depend
largely on ensuring that the needs of the users are considered the
primary goals of the design. Violating this principle may cause
disillusionment from the people who are volunteering their time to
assist in the project.
