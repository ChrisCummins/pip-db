\chapter{Evaluation}\label{chap:evaluation}

Evaluation of pip-db comprised of both qualitative and quantitative
components. Qualitative usability testing was conducted in mid April
at the end of the development, and a criteria-based quantitative
evaluation was performed at the start of May.

\section{Usability testing}\label{sec:usability-testing}

Usability tests were conducted over a ten day period, and consisted of
presenting a set of user scenarios to a participant which they would
work through. The scenario would involve a set of tasks designed to
test the intuitive user friendliness of the pip-db
website. Appendix~\ref{app:evaluation-script} describes the procedure
of the tests. Appendix~\ref{app:evaluation-scenarios} contains the
user scenarios which were evaluated. Five tests were conducted in
total, with the demographic of participants divided between PhD
students specialising in a relevant biology field and non-scientific
participants. In the case of user testing with participants who do not
have a scientific background, extra verbal explanation was given to
provide scientific context. Following recommended practises, each test
session was recorded using audio and screen capture of the testing
computer \cite{dumas1999practical}.

\subsubsection*{On sample size}
\cite{nielsen2000you, rubin2008handbook}

% TODO: Notes from testing

\section{Quantitative evaluation}\label{sec:quantitative-evaluation}

\cite{jackson2011evaluation}

% TODO: project goals

% TODO: Were risks mitigated
