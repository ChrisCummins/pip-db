\chapter{Conclusions}\label{chap:conclusions}

%% This is mandatory. The conclusion is the third most important
%% section of the document. Briefly restate the work done; summarise
%% any findings or recommendations emerging from the project. Everyone
%% who reads your report will read the conclusion.

The web application described in this document entirely fulfils the
stated goal of categorising and providing accessible online search
functionality for PIP-DB. In addition to satisfying the core
deliverables, a pragmatic approach to the development of
infrastructure and tooling has resulted in the creation of several
supporting projects, including the YAPS file format, pipbot repository
manager, plausible nonsense generator, and CSV analyser. Additionally,
contributions have been made to four popular existing open source
projects as a result of development: watch-less, clojure-koans,
sqlkorma, and gitstats.

While costly in terms of time, the radical switch in programming
language after the completion of the initial prototype allowed a rare
direct comparison to be made between non-trivial software written in
PHP and Clojure. The comparison found that the functional
implementation resulted in a 75\% reduction in code base size for
functionally identical web server implementations
\cite{cummins2014migrating}.

Development of an Autotooled build system for websites showed how
shell-level parallelism could be used to reduce execution times by a
factor of 5. Further work on the build system could reduce these times
further by enabling parallelism within the core of automake.

Further work on the pip-db website could include the development of
administrative tools for modifying data, and refinements to the
performance and security of the web server implementation. Increased
scalability could be provided by implementing multi-threading support
for request handling, and greater edge case handling could increase
the server's robustness. One overlooked aspect of pip-db is the user
account and registration system. The implementation of an accounts
system was left incomplete as time was devoted instead to the
implementation of novel features and user interaction refinements.

It my recommendation that further usability tests be performed on the
existing pip-db design before any modifications are made to the user
interface. While defended by some \cite{nielsen2000you}, it is
generally believed that a usability testing sample size of 5 is too
small to reveal an adequate number of usability problems
\cite{spool2001testing, woolrych2001and}. In addition to formal
usability testing, it would be possible to increase the amount of
tracking data which is collected about users by recording all of the
search queries are performed. This would provide quantitative data
which could be used to optimise the search engine for the most popular
queries.

The persistent storage component of pip-db could be the subject of
future research, with NoSQL technologies finding increasing usage for
the similar purposes of document storage \cite{tudorica2011comparison,
  mongo2014leading, hecht2011nosql, mongo2013top5}. Using a NoSQL
store in place of the existing PostgreSQL back-end would remove the
need for the vectorisation of YAPS records and greatly simply the data
upload logic.
