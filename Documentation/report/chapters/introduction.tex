\chapter{Introduction}\label{chap:introduction}

%% Introduction: This is mandatory. A description of the aims of the
%% work and an indication of how the work is presented in the report.
%% Typical target: a Chief Engineer who must have a comprehensive view
%% of the contents of the report from the introduction. Therefore it
%% should provide:

%% Context

%% Provide motivation for the project. Discuss the background to the
%% project, provide whatever background the reader will need in order
%% to understand your contribution, and including relevant previous
%% work and any appropriate literature. Describe any work needed to
%% establish detailed project requirements

%% Requirements

%% In the light of the “Context” section and any other relevant
%% factors, give a clear statement of the requirements that the
%% project’s end-product should meet. It should describe concisely
%% what your experiment, design or program should do, (the question of
%% how it is derived, or how it does it, will be dealt with in later
%% sections).

%% Overview

%% The last part of the introduction outlines the remainder of the
%% report, explaining what comes in each section.  The introduction is
%% the second most important section of the document. Everyone who
%% reads your report will read the introduction; some may read only
%% the Introduction and the Conclusion. Therefore, you should think
%% carefully about what you want to say, and in what order you should
%% say it. The introduction is also the most difficult one to write,
%% because it is hard to balance the requirements of giving adequate
%% explanation without entering into too much detail. For these
%% reasons, you should re-write the introduction when you’ve written
%% the rest of the Report (i.e. when you know what you have to
%% introduce).

% TODO: Placeholder
\lipsum[1-10]
