\chapter{Introduction}\label{chap:introduction}

%% Introduction: This is mandatory. A description of the aims of the
%% work and an indication of how the work is presented in the report.
%% Typical target: a Chief Engineer who must have a comprehensive view
%% of the contents of the report from the introduction. Therefore it
%% should provide:

%% Context

%% Provide motivation for the project. Discuss the background to the
%% project, provide whatever background the reader will need in order
%% to understand your contribution, and including relevant previous
%% work and any appropriate literature. Describe any work needed to
%% establish detailed project requirements

%% Requirements

%% In the light of the “Context” section and any other relevant
%% factors, give a clear statement of the requirements that the
%% project’s end-product should meet. It should describe concisely
%% what your experiment, design or program should do, (the question of
%% how it is derived, or how it does it, will be dealt with in later
%% sections).

%% Overview

%% The last part of the introduction outlines the remainder of the
%% report, explaining what comes in each section.  The introduction is
%% the second most important section of the document. Everyone who
%% reads your report will read the introduction; some may read only
%% the Introduction and the Conclusion. Therefore, you should think
%% carefully about what you want to say, and in what order you should
%% say it. The introduction is also the most difficult one to write,
%% because it is hard to balance the requirements of giving adequate
%% explanation without entering into too much detail. For these
%% reasons, you should re-write the introduction when you’ve written
%% the rest of the Report (i.e. when you know what you have to
%% introduce).


\subsection*{Submitted files}

A polylingual project Clojure LISP, JavaScript, Less CSS, M4sh, Make,
Python, and sh.

Documentation in \LaTeX and Markdown.


%%%%%%%%%%%%%%%%%%%%%%
%% Figure: dir-tree %%
%%%%%%%%%%%%%%%%%%%%%%
% TODO: Add cross references for each section!
\begin{figure}[H]
\centering
\begin{tikzpicture}[dirtree]
\node {\texttt{/}}
child {node {\texttt{bin} \textit{- Deployment build scripts}}}
child {node {\texttt{build} \textit{- Auxiliary build files and dependencies}}}
child {node {\texttt{Documentation} \textit{- Project documentation root}}
  child {node {\texttt{design}}
    child {node {\texttt{d1} \textit{- First iteration design documents}}}
    child {node {\texttt{d2} \textit{- Second iteration design documents}}}
 }
  child {node {\texttt{evaluation} \textit{- Usability evaluation documents}}}
  child {node {\texttt{midterm} \textit{- Mid-term report documents}}}
  child {node {\texttt{plan} \textit{- Source code for project plan}}}
  child {node {\texttt{report} \textit{- Source code for this document}}}
}
child {node {\texttt{extern} \textit{- External build dependencies}}}
child {node {\texttt{resources}}
  child {node {\texttt{css} \textit{- Less CSS style sheets}}}
  child {node {\texttt{fonts} \textit{- Font files}}}
  child {node {\texttt{img} \textit{- Website images}}}
  child {node {\texttt{js} \textit{- Client-side source code}}}
}
child {node {\texttt{scripts} \textit{- Auxiliary help scripts}}}
child {node {\texttt{src} \textit{- Server-side source code}}}
child {node {\texttt{test} \textit{- Unit tests for server-side code}}}
child {node {\texttt{tools} \textit{- Auxiliary tools}}
  child {node {\texttt{csv2yaps} \textit{- Dataset encoder}}}
  child {node {\texttt{watchr} \textit{- On-demand build system}}}
  child {node {\texttt{yaps2fsa} \textit{- FASTA database encoder}}}
};
\end{tikzpicture}
\caption[Project directory structure]
        {Project directory structure.}
\label{fig:dir-tree}
\end{figure}


% TODO: Note on file paths (relative vs absolute)

\newpage
\subsection*{Nomenclature}
Acronyms and initialisms will be used where appropriate in place of
full terms. When one is to be used, the first use of the term will
include the full expanded name, followed by the acronym in
parenthesis. The acronym will then be used from thereon in.


\subsubsection*{On the use of UML}
A subset of the Unified Modelling Language \cite{ibm2003uml} is used a
basis for many of the technical diagrams. One notable deviation from
convention is the use of sequence diagrams \cite{ibm2004sequence} to
describe the behaviour of web services. In such cases, the desired
effect is to illustrate the behaviour of a system at a high level, not
to provide a technically accurate description of communication.


\subsubsection*{On the project name}
Throughout this text it is necessary to carefully distinguish
conversation about the biological dataset from the software and
algorithms used to host it. To achieve this distinction, I will use
the initialism PIP-DB to refer to the biological dataset assembled by
members of the Life \& Health Sciences department, and the
\textit{name} pip-db to refer to the software project that I developed
to host this. In order to keep this distinction clear, the
capitalisation will be kept consistent irrespective of the grammatical
context. Thus, PIP-DB refers to a set of data, and pip-db refers to
the presentation logic for this dataset.


\subsection*{Overview}
The rest of the text is structured as follows: first a brief overview
of the requirements analysis phase of the project is given
(page~\pageref{chap:requirements}), with particular reference to the
project plan. Then, a large body of text is devoted to the design and
implementation of the final product, divided into three chapters:
process (page~\pageref{chap:process}), infrastructure
(page~\pageref{chap:infrastructure}), and product
(page~\pageref{chap:product}). The results of the project are
considered in the following evaluation chapter
(page~\pageref{chap:evaluation}), followed by the conclusions
(page~\pageref{chap:conclusions}). The appendices
(page~\pageref{appendices}) and bibliography
(page~\pageref{bibliography}) are included at the end.
