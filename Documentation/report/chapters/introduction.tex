\chapter{Introduction}\label{chap:introduction}

%% Introduction: This is mandatory. A description of the aims of the
%% work and an indication of how the work is presented in the report.
%% Typical target: a Chief Engineer who must have a comprehensive view
%% of the contents of the report from the introduction. Therefore it
%% should provide:

%% Context

%% Provide motivation for the project. Discuss the background to the
%% project, provide whatever background the reader will need in order
%% to understand your contribution, and including relevant previous
%% work and any appropriate literature. Describe any work needed to
%% establish detailed project requirements

\section*{Objectives}\label{sec:objectives}
The following objectives were written during before the planning stage
of the project, in order to provide a very high level overview of the
finished product. The purpose of the objectives it to provide personal
goals and expectations of the project, which can then be refined to
produce specific and quantifiable requirements
(Chapter~\ref{chap:requirements}).

\begin{enumerate}
\item To build a free (as in freedom) web application for searching
  and viewing protein isoelectric points.
\item To produce a bioinformatics tool with real world value for
  future scientific research.
\item The application should provide intuitive but powerful searching
  facilities.
\item The application should provide a convenient means for a
  certified user to edit and upload additional data.
\item The application should present information in a usable and
  efficient form.
\item Users should be allowed to download generated results for
  offline use.
\item Adequate security precautions should be taken to minimise the
  risk of data being sabotaged or stolen.
\item The implementation should use a clean model view controller
  architecture.
\item Comprehensive test coverage of the API and common use cases
  should be automated.
\item The application should be scalable for much larger datasets.
\end{enumerate}

%% Requirements

%% In the light of the “Context” section and any other relevant
%% factors, give a clear statement of the requirements that the
%% project’s end-product should meet. It should describe concisely
%% what your experiment, design or program should do, (the question of
%% how it is derived, or how it does it, will be dealt with in later
%% sections).


\newpage
\section*{Submission archive}
The accompanying archive for this report contains a snapshot of the
development repository. It was prepared automatically using the
\texttt{mksubmission.sh} script written for this purpose. While every
effort has been made to include relevant source code listings in this
report, it is intended that a curious reader will investigate the
contents of the archive of their own accord, as the large size of the
source code means that only a small proportion can be reproduced in
this report. Figure~\ref{fig:dir-tree} identifies the key directories
that may be of interest to the reader. The diagram shows a directory
tree, where each directory has been annotated with numbers
corresponding to the sections which discuss them within this
report. The UNIX path delimiter ``/'' is used throughout this text,
with absolute paths referring to the root directory of the archive.

This is a polylingual project including source code written in Clojure
LISP, JavaScript, Less CSS, M4sh, Make, Python, and sh programming
languages. The documentation is formatted in \LaTeX, HTML and
Markdown. A reasonably competent text editor is all that is required
to view the source files, which can be identified by the file
extensions: ac, am, bib, clj, fsa, in, js, json, less, md, py, sh,
tex, xml, yml.


%%%%%%%%%%%%%%%%%%%%%%
%% Figure: dir-tree %%
%%%%%%%%%%%%%%%%%%%%%%
\begin{figure}[H]
\begin{multicols}{2}
\centering
\textbf{Directory}

\raggedleft
% LEFT COLUMN: Directory tree
\begin{tikzpicture}[dirtree]
\node {\texttt{/}}
child {node {\texttt{bin}}}
child {node {\texttt{build}}}
child {node {\texttt{Documentation}}
  child {node {\texttt{design}}
    child {node {\texttt{d1}}}
    child {node {\texttt{d2}}}
  }
  child {node {\texttt{evaluation}}}
  child {node {\texttt{midterm}}}
  child {node {\texttt{plan}}}
  child {node {\texttt{report}}}
}
child {node {\texttt{extern}}}
child {node {\texttt{resources}}
  child {node {\texttt{css}}}
  child {node {\texttt{fonts}}}
  child {node {\texttt{img}}}
  child {node {\texttt{js}}}
}
child {node {\texttt{scripts}}}
child {node {\texttt{src}}}
child {node {\texttt{test}}}
child {node {\texttt{tools}}
  child {node {\texttt{csv2yaps}}}
  child {node {\texttt{watchr}}}
  child {node {\texttt{yaps2fsa}}}
};
\end{tikzpicture}
\columnbreak

\raggedright
\textbf{Report sections}

\textit{%
% RIGHT COLUMN: Section cross-references
\br{}\\ % /
\ref{sec:building}\\                       % /bin
\ref{sec:building}, \ref{sec:deployment}\\ % /build
\br{}\\                                    % /Documentation
\br{}\\                                    % /Documentation/design
\br{}\\                                    % /Documentation/design/d1
\br{}\\                                    % /Documentation/design/d2
\br{}\\                                    % /Documentation/evaluation
\br{}\\                                    % /Documentation/midterm
\br{}\\                                    % /Documentation/plan
\br{}\\                                    % /Documentation/report
\br{}\\                                    % /extern
\br{}\\                                    % /resources
\br{}\\                                    % /resources/css
\br{}\\                                    % /resources/fonts
\br{}\\                                    % /resources/img
\br{}\\                                    % /resources/js
\br{}\\                                    % /scripts
\br{}\\                                    % /src
\br{}\\                                    % /test
\br{}\\                                    % /tools/
\br{}\\                                    % /tools/csv2yaps
\br{}\\                                    % /tools/watchr
\br{}\\                                    % /tools/yaps2fsa
}
\end{multicols}
\caption[Archive directory structure]
        {Archive directory structure and report section cross-references.}
\label{fig:dir-tree}
\end{figure}

\newpage
\subsubsection*{Building the project}
A complete copy of the project can be compiled on a GNU/Linux
operating system by executing the command \texttt{./bin/build -{}-all}
from within the project root directory. This will invoke a script that
will attempt to automatically resolve system dependencies and
requirements. Failure to meet the system requirements will results in
an informative error message being displayed.

% TODO: Note on file paths (relative vs absolute)


\section*{Nomenclature}\label{sec:nomenclature}
Acronyms and initialisms will be used where appropriate in place of
full terms. When one is to be used, the first use of the term will
include the full expanded name, followed by the acronym in
parenthesis. The acronym will then be used from thereon in.


\subsubsection*{On the project name}
Throughout this text it is necessary to carefully distinguish
conversation about the biological dataset from the software and
algorithms used to host it. To achieve this distinction, I will use
the initialism PIP-DB to refer to the biological dataset assembled by
members of the Life \& Health Sciences department, and the
\textit{name} pip-db to refer to the software project that I developed
to host this. In order to keep this distinction clear, the
capitalisation will be kept consistent irrespective of the grammatical
context. Thus, PIP-DB refers to a set of data, and pip-db refers to
the presentation logic for this dataset.


\subsubsection*{On the use of UML}
A subset of the Unified Modelling Language \cite{ibm2003uml} is used a
basis for many of the technical diagrams. One notable deviation from
convention is the use of sequence diagrams \cite{ibm2004sequence} to
describe the behaviour of web services. In such cases, the desired
effect is to illustrate the behaviour of a system at a high level, not
to provide a technically accurate description of communication.


\section*{Overview}\label{sec:overview}
%% The last part of the introduction outlines the remainder of the
%% report, explaining what comes in each section.  The introduction is
%% the second most important section of the document. Everyone who
%% reads your report will read the introduction; some may read only
%% the Introduction and the Conclusion. Therefore, you should think
%% carefully about what you want to say, and in what order you should
%% say it. The introduction is also the most difficult one to write,
%% because it is hard to balance the requirements of giving adequate
%% explanation without entering into too much detail. For these
%% reasons, you should re-write the introduction when you’ve written
%% the rest of the Report (i.e. when you know what you have to
%% introduce).
The rest of the text is structured as follows: first a brief overview
of the requirements analysis phase of the project is given
(page~\pageref{chap:requirements}), with particular reference to the
project plan. Then, a large body of text is devoted to the design and
implementation of the final product, divided into three chapters:
process (page~\pageref{chap:process}), infrastructure
(page~\pageref{chap:infrastructure}), and product
(page~\pageref{chap:product}). The results of the project are
considered in the following evaluation chapter
(page~\pageref{chap:evaluation}), followed by the conclusions
(page~\pageref{chap:conclusions}). The appendices
(page~\pageref{appendices}) and bibliography
(page~\pageref{bibliography}) are included at the end.
