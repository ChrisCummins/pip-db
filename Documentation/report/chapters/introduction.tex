\chapter{Introduction}\label{chap:introduction}

%% Introduction: This is mandatory. A description of the aims of the
%% work and an indication of how the work is presented in the report.
%% Typical target: a Chief Engineer who must have a comprehensive view
%% of the contents of the report from the introduction. Therefore it
%% should provide:

%% Context

%% Provide motivation for the project. Discuss the background to the
%% project, provide whatever background the reader will need in order
%% to understand your contribution, and including relevant previous
%% work and any appropriate literature. Describe any work needed to
%% establish detailed project requirements

%% Requirements

%% In the light of the “Context” section and any other relevant
%% factors, give a clear statement of the requirements that the
%% project’s end-product should meet. It should describe concisely
%% what your experiment, design or program should do, (the question of
%% how it is derived, or how it does it, will be dealt with in later
%% sections).

%% Overview

%% The last part of the introduction outlines the remainder of the
%% report, explaining what comes in each section.  The introduction is
%% the second most important section of the document. Everyone who
%% reads your report will read the introduction; some may read only
%% the Introduction and the Conclusion. Therefore, you should think
%% carefully about what you want to say, and in what order you should
%% say it. The introduction is also the most difficult one to write,
%% because it is hard to balance the requirements of giving adequate
%% explanation without entering into too much detail. For these
%% reasons, you should re-write the introduction when you’ve written
%% the rest of the Report (i.e. when you know what you have to
%% introduce).


\subsection*{Submitted files}

A polylingual project Clojure LISP, JavaScript, Less CSS, M4sh, Make,
Python, and sh.

Documentation in \LaTeX and Markdown.


%%%%%%%%%%%%%%%%%%%%%%
%% Figure: dir-tree %%
%%%%%%%%%%%%%%%%%%%%%%
% TODO: Add cross references for each section!
\begin{figure}[H]
\centering
\begin{tikzpicture}[dirtree]
\node {\texttt{/}}
child {node {\texttt{bin} \textit{- Deployment build scripts}}}
child {node {\texttt{build} \textit{- Auxiliary build files and dependencies}}}
child {node {\texttt{Documentation} \textit{- Project documentation root}}
  child {node {\texttt{design}}
    child {node {\texttt{d1} \textit{- First iteration design documents}}}
    child {node {\texttt{d2} \textit{- Second iteration design documents}}}
 }
  child {node {\texttt{evaluation} \textit{- Usability evaluation documents}}}
  child {node {\texttt{midterm} \textit{- Mid-term report documents}}}
  child {node {\texttt{plan} \textit{- Source code for project plan}}}
  child {node {\texttt{report} \textit{- Source code for this document}}}
}
child {node {\texttt{extern} \textit{- External build dependencies}}}
child {node {\texttt{resources}}
  child {node {\texttt{css} \textit{- Less CSS style sheets}}}
  child {node {\texttt{fonts} \textit{- Font files}}}
  child {node {\texttt{img} \textit{- Website images}}}
  child {node {\texttt{js} \textit{- Client-side source code}}}
}
child {node {\texttt{scripts} \textit{- Auxiliary help scripts}}}
child {node {\texttt{src} \textit{- Server-side source code}}}
child {node {\texttt{test} \textit{- Unit tests for server-side code}}}
child {node {\texttt{tools} \textit{- Auxiliary tools}}
  child {node {\texttt{csv2yaps} \textit{- Dataset encoder}}}
  child {node {\texttt{watchr} \textit{- On-demand build system}}}
  child {node {\texttt{yaps2fsa} \textit{- FASTA database encoder}}}
};
\end{tikzpicture}
\caption[Project directory structure]
        {Project directory structure.}
\label{fig:dir-tree}
\end{figure}


% TODO: Note on file paths (relative vs absolute)


\subsection*{On the use of diagrams}

% UML DIAGRAMS:

\cite{ibm2003uml}

% UML Sequence diagrams:

\cite{ibm2004sequence, ibm2006wsr}


\subsection*{Nomenclature}

\subsection*{Overview}
