\chapter{Process}\label{chap:process}

The following chapter describes the processes used in the design and
implementation of the project. Section~\ref{sec:development-process}
provides an explanation of the development process and the approach to
implementation, and Section~\ref{sec:design-process} describes the
user-centred aspects of the design process.

\section{Development process}\label{sec:development-process}

A project with an individual developer requires a different approach
to time management than a multi-developer project. The lack of other
team members means that the development can afford to take a much more
flexible and dynamic approach, allowing for a faster pace and lower
cost of change in the development life cycle. Agile software processes
focus on this fast pace of change by encouraging frequent
communication with stakeholders and very short development iterations
\cite{highsmith2001agile, martin2003agile}.

Elements of the Agile manifesto \cite{fowler2001agile} inspired
decisions in the project development. For example, frequent meetings
with Dr Flower were used to provide ongoing feedback of development
progress, and meets the agile requirement for \textit{customer
  collaboration over contract negotiation}. Additionally, the
requirement for \textit{working software over comprehensive
  documentation} was used to justify the early development of a
working prototype which users could interact with, instead of lengthy
requirements notifications with potential users before beginning
development work. This resulted in useful design feedback during early
stages of the project development, as it is more intuitive for users
to provide feedback for a functional prototype then it is to discuss
requirements in a more abstract manner without being able to interact
with a product.

Not all of the Agile software philosophy was strictly adhered to. In
particular, the emphasis on \textit{responding to change over
  following a plan} was supplanted by the requirement for a formal
project plan document to be created in the first term. Since the Agile
philosophy doesn't provide a template or process for guiding
development, so the iterative OpenUP development process was used to
provide a time management framework, by dividing the development life
cycle into discrete increments, each consisting of an inception,
elaboration, construction, and evaluation phase
\cite{balduino2007introduction}. OpenUP was chosen for the development
process due to its entirely open source nature as part of the Eclipse
Process Framework, and because it targets small teams and agile
development by design \cite{kroll2006agility}. Development was split
over three iterations, with one covering the first term, and two in
the second term. The end of each iteration's evaluation phase
culminated in a design and implementation milestone pair.

\subsection{Version Control}\label{subsec:version-control}

The git version control system was used to provide version control. A
single monolithic git repository tracks revisions for all pip-db
source codes and associated data. By using version control from the
very project's inception, a fully accountable and transparent history
of development has been recorded, with 2,420 revisions committed since
14 October 2013.

Git was chosen as the version control system due to its support for
lightweight branching and distributed-by-design nature
\cite{chacon2009pro}. While the benefits of distributed version
control are not entirely exploited for single-developer projects, the
support for concurrent development of branches encourages
experimentation and an agile approach to development.

\subsection{Open Source}\label{subsec:open-source}

One of the key considerations of the project objectives
(Chapter~\ref{sec:objectives}) was that the finished project should be
freely available without commercial interest, and this extends to the
source code and development process. The success of truly open models
of development have been investigated in great detail
\cite{weber2004success, godfrey2000evolution, chesbrough2006open,
  von2005democratizing}, and a philosophy of ``release early, release
often'' has been encouraged in open source communities as a technique
for nurturing rapid and widespread user involvement from an early
stage \cite{raymond1999cathedral}. As a result, all of the program
code, documentation and supporting files that have been created for
pip-db have been released under the terms of the GNU General Public
License v3. This is an open and permissive license that allows for
commercial use, but it mandates that derivative works maintain the
same license and must distribute the source code openly
\cite{gnu2007gpl}.

The combination of an open source license and the use of git version
control meant that online repository hosting could be used to provide
a public centre for development. For this, the GitHub website was
used, which is the most popular online repository hosting site
\cite{finley2011github}, and offers free hosting for open source
projects \cite{cummins2014pip-db}.


%%%%%%%%%%%%%%%%%%%%%%%%%%%%
%% Figure: github-project %%
%%%%%%%%%%%%%%%%%%%%%%%%%%%%
\begin{figure}[H]
\centering
    \includegraphics[width=0.82\textwidth]{assets/github}
\caption[Screenshot of GitHub project homepage]
        {Screenshot of the GitHub repository for pip-db.}
\label{fig:github-project}
\end{figure}


\subsubsection*{A note on dataset confidentiality}
It is important to note that while pip-db is an open source project,
the PIP-DB dataset as supplied by Dr Flower remains confidential at
his request, and so has not been released for distribution.

\subsection{Development workflow}\label{subsec:github-workflow}

In addition to hosting the project source code and revision history,
GitHub provides many useful features intended for collaborative
development efforts, including an issue tracker, milestones, and a
Wiki. By combining these features with strict version control
practises, it is possible to create a dynamic development environment
which simultaneously encourages experimentation and rapid change while
providing a full history of revisions and the ability to roll back and
integrate new features when required.

The issue tracker provided by GitHub was used from the project
inception. The purpose of the issue tracker is to document requests
for changes. Whilst the revision control log provides a history of all
of the changes which have been made, the issue tracker is used as a
place to document changes which \textit{should} be made, but not have
not yet been completed. Each issue is assigned a unique identifier,
and these identifiers can be used in revision messages to provide a
cross reference between the revision control history and the known
issues and bugs. In total, 350 issues have been opened, of which 39
remain open at the time of writing.

Issues can be categorised using labels
(Table~\ref{tab:issue-labels}). Labels provide additional meta data
regarding a type of issue, and each issue can be assigned multiple
labels. The issue tracker can filter issues by labels, allowing for a
quick visual overview of the issues particular types.


%%%%%%%%%%%%%%%%%%%%%%%%%%%
%% Figure: github-issues %%
%%%%%%%%%%%%%%%%%%%%%%%%%%%
\begin{figure}[H]
\centering
    \includegraphics[width=0.82\textwidth]{assets/github-issues}
\caption[Screenshot of GitHub's pip-db issue tracker]
        {Screenshot of GitHub's issue tracker for pip-db.}
\label{fig:github-issues}
\end{figure}


%%%%%%%%%%%%%%%%%%%%%%%%%
%% Table: issue-labels %%
%%%%%%%%%%%%%%%%%%%%%%%%%
\begin{table}[H]
\centering
\begin{tabular}{l l}
\textbf{Label} & \textbf{Description}\\
\hline
Bug & Crash reports, stack traces, and software failures.\\
Design & Issues relating to the user interface design.\\
Documentation & Documentation tasks.\\
Feature & Web service feature addition requests.\\
Implementation & Issues relating to the web server implementation.\\
Task & Feature addition requests.\\
Testing \& tooling & Issues relating to infrastructure.\\
Build system & Bugs and issues with the build system.\\
Planning & Issues relating to TP1 project planning.\\
Regression & Issues which have arisen as a result of changes, not additions.\\
Version control & Git and GitHub issues and feature requests.\\
Wontfix & Used to indicate issues which have been closed without being fixed.\\
\end{tabular}
\caption[Issue tracker labels]
        {The labels used for categorising issues, and their corresponding meanings.}
\label{tab:issue-labels}
\end{table}


In addition to assigning labels to issues, GitHub also supports the
creation of milestones with set dates. Issues can be assigned to
milestones, and the number of open and closed issues per milestone can
be shown, providing an indication of the progress towards a particular
milestones (Figure~\ref{fig:github-milestones}).


%%%%%%%%%%%%%%%%%%%%%%%%%%%%%%%
%% Figure: github-milestones %%
%%%%%%%%%%%%%%%%%%%%%%%%%%%%%%%
\begin{figure}[H]
\centering
    \includegraphics[width=0.82\textwidth]{assets/github-milestones}
\caption[Screenshot of open project milestones]
        {Screenshot of GitHub's milestones overview for pip-db.}
\label{fig:github-milestones}
\end{figure}


\subsection{Branching model}\label{subsec:branching-model}

A branching model was designed for this project in order to provide a
consistent branching and release strategy to use over the course of
development. Since the revision control must track a huge number of
changes (over 2000) over the course of several months, it is important
that the revision history be as clear as possible, and that branches
are used intelligently to provide additional information about the
project development, not to obscure past work.

The branch model that I designed was inspired by Driessen's
\textit{Successful Git branching model} \cite{driessen2012successful},
with a number of changes to adapt it specifically for this
project. The core of Driessen's model is two permanent branches which
track the current development head, and the latest stable
release. Developers work on the development branch, and update the
stable branch at release time. Transient auxiliary branches are used
as staging areas for new features and releases. The hotfix branch
support was removed from Driesson's model, since the software
developed is only proof and concept and so does not need to support
regression patching.


%%%%%%%%%%%%%%%%%%%%%%%%%
%% Table: branch-names %%
%%%%%%%%%%%%%%%%%%%%%%%%%
\begin{table}[H]
\centering
\begin{tabular}{l l L{8cm}}
\textbf{Driesson} & \textbf{Cummins} & \textbf{Purpose}\\
\hline
\texttt{master} & \texttt{stable} & The latest stable release.\\
\texttt{develop} & \texttt{master} & The current development head.\\
\texttt{release/:name} & \texttt{release/:version} & Release candidate staging areas.\\
\texttt{feature/:name} & \texttt{wip/:id} & Feature addition branches (cross referenced with issue tracker using issue IDs).\\
\texttt{hotfix/:name} & & Hotfix development branches.\\
\end{tabular}
\caption[Development model branch names]
        {A comparison of branch names with Driesson's development model.}
\label{tab:branch-names}
\end{table}


Importantly, the name scheme for feature branches was changed so that
it matched the GitHub issue IDs. This meant that work on a feature
should only begin when it has an open issue assigned to it, enforcing
the use of the issue tracker. This novel integration of issue tracker
and version control system means that for every change made in the
project, it is possible to trace back not only the revision which
introduced the change, but also the issue or feature request which
demanded it. This means that every change is justified with a reason
\textit{why} the change was made, not just the description of
\textit{how} the change was made which is provided by the revision
history.

\subsection{Auxiliary documentation}\label{subsec:aux-docs}

To ensure a consistent use of version control, a checklist was created
for each common activity (submitting a patch, creating a release,
starting work on a new feature). Additional files document the
high-level workflow and approach to release management. Relevant
documentation in the submission archive includes:

\begin{verbatim}
    Documentation/ReleaseChecklist.html
    Documentation/SubmitChecklist.html
    Documentation/SubmittingPatches.html
    Documentation/Workflow.html
    Documentation/VersionNumbering.html
\end{verbatim}

An engineer's log was used to keep daily notes of all development
activity, minutes from stakeholder meetings, and other tertiary
information that is missing from the revision history and issue
tracker. A HTML render of the log can be found in the submission
archive \texttt{Documentation/Log.html}. The project log was written
in Markdown format for quick typesetting, and used a consistent date
format to separate individual entries which meant it was possible to
parse the log using a simple Python script
(\texttt{scripts/parselog.py}) for export into different formats and
analysis (Table~\ref{tab:log-parse}).


%%%%%%%%%%%%%%%%%%%%%%
%% Table: log-parse %%
%%%%%%%%%%%%%%%%%%%%%%
\begin{table}[H]
\centering
\begin{tabular}{R{8cm} L{8cm}}
Number of log entries & 127\\
Total word count & 21,329\\
Average entry word count & 167\\
Shortest entry word count & 6\\
Longest entry word count & 2,008\\
\end{tabular}
\caption[Project log details]
        {Project log details.}
\label{tab:log-parse}
\end{table}


\newpage
\section{Design process}\label{sec:design-process}

The complement of the development processes are the design processes,
which dictate the approach to user interface design. Both the project
objectives and risk mitigation strategies emphasise the importance of
creating an intuitive user interface for the search engine, making the
decisions in design processes important to the success of the project.

\subsection{Human-centred design}\label{subsec:user-centred-design}

The software industry is undoing a renaissance in its attitude towards
usability. In the post text-interface age, the graphical user
interface has become the focus of many interesting shifts in design
methodology. Human-centred design is a school of interaction design
which edifies \textit{usability} as the primary goal of all design
\cite{maguire2001methods}, and encourages designers to take a measured
approach to interaction design, since ``the joy of an early release
lasts but a short time. The bitterness of an unusable system lasts for
years''.

Uptake of human-centred design principles has been slower in the field
of bioinformatics, which may be due to the common belief that
usability is only of secondary importance to functionality, instead of
being a core component of it. The disadvantage of this is that a
number of popular existing bioinformatics tools have relatively poor
usability, although there is evidence of attempts at innovation in
some of the more popular tools \cite{lu2011pubmed,
  hearst2007biotext}. There have been efforts made to introduce
human-centred design into bioinformatics, with studies showing that
``although users believe that the bioinformatics community is
providing accurate and valuable data, they often find the interfaces
to these resources tricky to use and navigate''
\cite{pavelin2012bioinformatics}. The importance of good usability as
a time saving mechanism has been justified as ``usability `barriers'
can pose significant obstacles to a satisfactory user experience and
force researchers to spend unnecessary time and effort to complete
their tasks'' \cite{bolchini2009better}.

Over the course of the project development, a significant emphasis was
placed on taking a human-centred approach design, not only because of
the time saving benefits of a well designed tool, but also as a
necessary coping mechanism for the difference in usability
expectations between the computer science and biochemistry
communities. Further risk mitigation is provided by extensive analysis
of existing bioinformatics tools in the project planning
stage\footnote{See the project plan for a critical analysis of
  existing bioinformatics search engines. A copy of the project plan
  can be found in the submission archive
  \texttt{Documentation/ProjectPlan.pdf}.}.

\subsection{Prototype development}\label{subsec:prototype-development}

Rapid prototyping played a key role in being able to achieve the goal
of human-centred design, allowing for immediate and visual feedback
from stakeholders and potential users. The project used two types of
prototyping: low fidelity and high fidelity.

\subsubsection*{Low fidelity prototyping}

Low fidelity prototyping involves the rapid generation of
non-functional ``paper prototypes'' which are designed to give a rough
impression of how the finished product will look, without specifying
an implementation design. Low fidelity prototyping was achieved in
this project using the Balsamiq\footnote{Balsamiq. Rapid, effective
  and fun wireframing software. \url{http://balsamiq.com/}} program to
generate mockups and wireframes of the pages of the website
(Figure~\ref{fig:d1-mockups}). These mockups were discussed with Dr
Flower to verify that they were broadly satisfactory, and then used as
the basis for the M1 implementation. After the M1 implementation was
complete, the mockups were updated and refined for the second D2
design milestone (Figure~\ref{fig:d2-mockups}).


%%%%%%%%%%%%%%%%%%%%%%%%%%%%
%% Figure: github-project %%
%%%%%%%%%%%%%%%%%%%%%%%%%%%%
\begin{figure}[H]
\centering
    \includegraphics[width=\textwidth]{assets/d1-mockups}
\caption[D1 design mockups for site pages]
        {D1 design mockups for site pages.}
\label{fig:d1-mockups}
\end{figure}


%%%%%%%%%%%%%%%%%%%%%%%%
%% Figure: d2-mockups %%
%%%%%%%%%%%%%%%%%%%%%%%%
\begin{figure}[H]
\centering
    \includegraphics[width=\textwidth]{assets/d2-mockups}
\caption[D2 design mockups for site pages]
        {D2 design mockups for site pages.}
\label{fig:d2-mockups}
\end{figure}

\newpage
In addition to providing a guideline for the page aesthetics, the
mockups were used to give a rough outline of the interaction design. A
list of common tasks was created and for each, a sequence of mockups
was generated which show the steps that the user would have to take to
achieve them. Figure~\ref{fig:d1-search} shows the interaction design
for a simple name search using the D1 design. Successive design
iterations refined these interaction designs further, and covered a
wider variety of use cases, such as exception handling and error
cases.


%%%%%%%%%%%%%%%%%%%%%%%
%% Figure: d1-search %%
%%%%%%%%%%%%%%%%%%%%%%%
\begin{figure}[H]
\centering
    \includegraphics[width=\textwidth]{assets/d1-search}
\caption[D1 interaction design for a simple use case]
        {D1 interaction design for a simple use case.}
\label{fig:d1-search}
\end{figure}


\subsubsection*{High fidelity prototyping}

In contrast to low fidelity prototypes, the purpose of a high fidelity
prototype is to provide a way for potential users and stakeholders to
interact with a tangible simulation of the final product
\cite{egger2000lofi}. This depth first prototyping requires a more
technical approach which maps the high level design ideas of the low
fidelity prototype to a functional back-end to provide dynamic
behaviour.

In order to mitigate the risk of creating a tool which is not
intuitive, an initial prototype implementation was required for the M1
milestone, at the end of the first term. This meant that the second
term development could be focused upon refined and innovation of the
user interface. There was a tight time schedule for development of
this first prototype, so an opportunistic approach to programming was
adopted, making maximum use of existing tools and supporting software
where available \cite{brandt2008opportunistic}. The implementation of
the M1 prototype is discussed in
Section~\ref{sec:prototype-implementation}. See
Appendix~\ref{app:d1-m1-comparison} for a comparison of the D1 low
fidelity mockups with the M1 prototype implementation.
