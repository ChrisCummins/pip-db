\chapter{Risk Assessment}\label{chap:risks}

It is expected that 400 hours of work be put into a final year
project. With such a large body of time dedicated to it, it is
important to establish a clear direction for development, such that
the project progress can be assessed at regular intervals along the
project life cycle. The aim of progress assessments is to ensure that
time is not wasted on work which does not positively contribute to the
development of a finished product. In single-developer projects there
is an even greater priority for regular progress assessments than in a
large team project, as the single-developer is not accountable to
anyone, leading to a greater chance of the project losing focus or
suffering from second-system effect \cite{brooks1995mythical}.

To reduce the chance of this, a set of project risks were identified
during the planning stage (Table~\ref{tab:risks}), and development was
focused around the early mitigation of these risks. For each risk, the
probability of it occurring and the impact that it would have on
project progress was assigned a numerical value between 0-5, allowing
the risks to ordered in terms of severity using the geometric mean of
these two values. High level risk mitigation strategies were
constructed (Appendix~\ref{app:risk-mitigation}), and used as a
starting point for creating the project life cycle plan.

% TODO: justify and lead into process chapter


%%%%%%%%%%%%%%%%%%
%% Table: risks %%
%%%%%%%%%%%%%%%%%%
\begin{table}[H]
\centering
\begin{tabular}{c l l c c}
\textbf{Risk} & \textbf{Description}                              & \textbf{Category}     & \textbf{P} & \textbf{I}\\
\hline
\textbf{R1}   & Design is not intuitive                           & \textit{Design}       & 2          & 3\\
\textbf{R2}   & Project involves use of new technical skills      & \textit{Development}  & 5          & 5\\
\textbf{R3}   & High Level of technical complexity                & \textit{Development}  & 5          & 3\\
\textbf{R4}   & Complex deployment of production website          & \textit{Development}  & 5          & 4\\
\textbf{R5}   & Project milestones not clearly defined            & \textit{Planning}     & 1          & 1\\
\textbf{R6}   & System requirements not adequately identified     & \textit{Requirements} & 2          & 5\\
\textbf{R7}   & Change in project requirements during development & \textit{Requirements} & 1          & 5\\
\textbf{R8}   & Changes in dataset format during development      & \textit{Resources}    & 2          & 5\\
\textbf{R9}   & Unable to obtain required resources               & \textit{Resources}    & 1          & 1\\
\textbf{R10}  & Users not committed to the project                & \textit{Users}        & 2          & 4\\
\textbf{R11}  & Lack of cooperation from users                    & \textit{Users}        & 1          & 4\\
\textbf{R12}  & Users with negative attitudes toward the project  & \textit{Users}        & 1          & 2\\
\end{tabular}
\caption[Project risks]
        {Project risks. The \textbf{P} and \textbf{I} columns assign
         numerical values to the each risk's probability and impact
         respectively, within the range 0-5.}
\label{tab:risks}
\end{table}
