%%%%%%%%%%%%%%%%%%%%%%%%%
%% Document and Layout %%
%%%%%%%%%%%%%%%%%%%%%%%%%

\documentclass[12pt]{report}

\usepackage[utf8]{inputenc}

% Make internal macro definitions accessible,
% e.g. \@title, \@date \@author.
\makeatletter

% Define the document margins.
\newcommand{\DocumentHorizontalMargin}{25mm}
\newcommand{\DocumentVerticalMargin}{25mm}

% Geometry package configuration. Sets page margins.
\usepackage[a4paper,
            left   = \DocumentHorizontalMargin{},
            right  = \DocumentHorizontalMargin{},
            top    = \DocumentVerticalMargin{},
            bottom = \DocumentVerticalMargin{}]{geometry}

% Fancy headers.
\usepackage{fancyhdr}
\fancyhead{}
\fancyhead[L]{\leftmark}
\fancyhead[R]{\rightmark}
\fancyfoot{}
\fancyfoot[C]{\thepage}
\headheight 15pt
\pagestyle{fancy}

% Multi-column support.
\usepackage{multicol}
% We remove any separation between columns:
\setlength{\columnsep}{0pt}


%%%%%%%%%%%%%%%%%%%%%%%
%% Table of Contents %%
%%%%%%%%%%%%%%%%%%%%%%%

\usepackage[dotinlabels]{titletoc}

% The tocbibind package can be used to add document elements like a
% bibliography or an index to the Table of Contents. Options:
%
%  notlof - Don't add List Of Figures to TOC
%  notlot - Don't add List Of Tables to TOC
%
\usepackage[nottoc,notlof,notlot]{tocbibind}

% Control the table of contents nested depth. Levels are:
%
%     0 - Chapter
%     1 - Section
%     2 - Subsection
%     3 - Subsubsection
%
\setcounter{tocdepth}{2}


%%%%%%%%%%%%%%%%%%%%%%%%%%%%%%%%%
%% Bibliography and Appendices %%
%%%%%%%%%%%%%%%%%%%%%%%%%%%%%%%%%

% Sort references by the order in which they are cited
\usepackage[style=numeric-comp,sorting=none]{biblatex}
\addbibresource{ref.bib}

% Appendix package. Documentation:
%
%  http://mirror.ox.ac.uk/sites/ctan.org/macros/latex/contrib/appendix/appendix.pdf
%
% Package options:
%
% toc      - Put a header (e.g., `Appendices') into the Table of Contents
%            (the ToC) before listing the appendices. (This is done by
%            calling the \addappheadtotoc command.)
% page     - Puts a title (e.g., `Appendices') into the document at the
%            point where the appendices environment is begun. (This is
%            done by calling the \appendixpage command.)
% title    - Adds a name (e.g., `Appendix') before each appendix title in
%            the body of the document. The name is given by the value
%            of \appendixname. Note that this is the default behaviour
%            for classes that have chapters.
% titletoc - Adds a name (e.g., `Appendix') before each appendix listed
%            in the ToC. The name is given by the value
%            of \appendixname.
% header   - Adds a name (e.g., `Appendix') before each appendix in page
%            headers.  The name is given by the value
%            of \appendixname. Note that this is the default behaviour
%            for classes that have chapters.
\usepackage[titletoc]{appendix}


%%%%%%%%%%%%%%%%%%%%%%%%%%%%%%%%%%%%%
%% Figures, footnotes and listings %%
%%%%%%%%%%%%%%%%%%%%%%%%%%%%%%%%%%%%%

\usepackage{float}
\restylefloat{figure}

% Use arabic numbers for footnote.
\renewcommand{\thefootnote}{\arabic{footnote}}

% Pre-requisites for rendering upquotes in listings package.
\usepackage[T1]{fontenc}
\usepackage{lmodern}
\usepackage{textcomp}

% Code listings.
\usepackage{listings}

\lstset{xleftmargin=0.05\textwidth,  % Don't use the full page width
        xrightmargin=0.05\textwidth,
        frame=bt,                    % Add top and bottom frame lines
        captionpos=b,                % Place caption below listing
        numbers=left,                % Add left-side line numbers
        basicstyle=\footnotesize,    % Use a small font size
        numberstyle=\footnotesize\bfseries,
        upquote=true,                % Use upright quotes, not curly
        commentstyle=\bfseries}      % Embolden comments

%%%%%%%%%%%%%%%%%%%%%%%%
%% Graphics and maths %%
%%%%%%%%%%%%%%%%%%%%%%%%

\usepackage{graphicx}
\usepackage{mathtools}
\usepackage{tikz}
\usepackage{tikz-qtree}

% Tikz flowchart configuration.
\usetikzlibrary{shapes,arrows}
\tikzstyle{decision} = [diamond,
                        draw,
                        text width=4.5em,
                        text badly centered,
                        node distance=3cm,
                        inner sep=0pt]
\tikzstyle{block}    = [rectangle,
                        draw,
                        text width=5em,
                        text centered,
                        node distance=3cm,
                        rounded corners,
                        minimum height=4em]
\tikzstyle{line}     = [draw, -latex']


%%%%%%%%%%%%%%%%%%%%%%
%% Tables and lists %%
%%%%%%%%%%%%%%%%%%%%%%

\usepackage{enumitem}
\setenumerate{itemsep=0pt}

\usepackage{longtable}


%%%%%%%%%%%%%%%%%%%%%%%%%%%%%
%% Typesetting and symbols %%
%%%%%%%%%%%%%%%%%%%%%%%%%%%%%

% Define a command to allow word breaking.
\newcommand*\wrapletters[1]{\wr@pletters#1\@nil}
\def\wr@pletters#1#2\@nil{#1\allowbreak\if&#2&\else\wr@pletters#2\@nil\fi}

% Define a command to create centred page titles.
\newcommand{\centredtitle}[1]{
  \begin{center}
    \large
    \vspace{0.9cm}
    \textbf{#1}
  \end{center}}

% Support hyperlinks using the \hyperref, \url and \href
% macros. Usage:
%
%    \hyperref[label_name]{''link text''}
%
%    \url{<my_url>}
%
%    \href{<my_url>}{<description>}
%
\usepackage{hyperref}

% Provide generic commands \degree, \celsius, \perthousand, \micro
% and \ohm which work both in text and maths mode.
\usepackage{gensymb}


%%%%%%%%%%%%%%%%%%%%%%%%%%%%%%%%%
%% Placeholder text generation %%
%%%%%%%%%%%%%%%%%%%%%%%%%%%%%%%%%

% Use either \blindtext or \libpsum to generate placeholder text. Also
% note the macros \blinditemize, \blindenumerate, \blinddescription.
\usepackage[english]{babel}
\usepackage{blindtext}
\usepackage{lipsum}
