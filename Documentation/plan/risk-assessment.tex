\newpage
\section{RISK ASSESSMENT \hrulefill}

Table \ref{tab:risk-assessment} lists some of the potential project risks that
were identified during the initial research phase which could influence the
success of the project and its ability to meet the objectives and
deliverables. For each risk, the probability of it occurring and impact it would
have on the project have been assigned a value between 1 and 5 to indicate their
magnitude.

\begin{table}[H]
\centering
\begin{tabular}{ | l | l | l || c | c | }
\hline
Risk & Description & Category & Probability & Impact\\
\hline
R1  & Design is not intuitive                           & Design       & 2 & 3\\
R2  & Project involves use of new technical skills      & Development  & 5 & 5\\
R3  & High Level of technical complexity                & Development  & 5 & 3\\
R4  & Complex deployment of production website          & Development  & 5 & 4\\
R5  & Project milestones not clearly defined            & Planning     & 1 & 1\\
R6  & System requirements not adequately identified     & Requirements & 2 & 5\\
R7  & Change in project requirements during development & Requirements & 1 & 5\\
R8  & Changes in dataset format during development      & Resources    & 2 & 5\\
R9  & Unable to obtain required resources               & Resources    & 1 & 1\\
R10 & Users not committed to the project                & Users        & 2 & 4\\
R11 & Lack of cooperation from users                    & Users        & 1 & 4\\
R12 & Users with negative attitudes toward the project  & Users        & 1 & 2\\
\hline
\end{tabular}
\caption{A list of potential project risks and their severity}
\label{tab:risk-assessment}
\end{table}

\subsection{Mitigation Strategies}

For each of the risks discovered in the assessment, mitigation strategies have
been defined which provide techniques to avoid or minimise the threat of each
risk.

\begin{longtable}{ | p{1cm} | p{6cm} | p{10cm} | }
  \hline
  \textbf{Risks} & \textbf{Description} & \textbf{Mitigation Strategy}\\
  \hline

\endfirsthead
  \hline
  \textbf{Risks} & \textbf{Description} & \textbf{Mitigation Strategy}\\
  \hline

\endhead
  \hline
  \multicolumn{3}{r}{\emph{Continued on next page}}
\endfoot

\endlastfoot
R1 & Design is not intuitive & The key to mitigation of this risk is in frequent
and effective user testing and an understanding of typical and common use-cases
for the product.\\

R2 & Project involves use of new technical skills & In order to prevent this
risk from having a serious impact on the project, it will be necessary to begin
studying and reading about the technologies that will be used at a very early
stage in the project, long before the start of the implementation.\\

R3 & High Level of technical complexity & Avoiding this risk will involve
ensuring that the scope of the project remains technically feasible, and that
the software architecture is abstracted into small enough units that it is
easier to focus on each one separately, as well as keeping small iterative
development cycles and adequate test coverage to prevent regressions when
implementing new functionality.\\

R4 & Complex deployment of production website & A website with independent data
and application logic components can result in an intricate deployment
process. This is a common problem in the development of complex web application,
where development and production environments must be synchronised and
differences between debugging and releases builds must be accounted for. In
order to mitigate this risk, a suite of tools to configure, build and deploy the
website should be developed at an early stage, allowing for fast deployment of
public releases.\\

R5 & Project milestones not clearly defined & A thoroughly described and well
thought out project plan will help to prevent scheduling issues and delays in
development that would arise from this risk.\\

R6 & System requirements not adequately identified & A comprehensive
specification of the finished product before implementation begins will help to
mitigate this risk.\\

R7 & Change in project requirements during development & An agile approach
towards accommodating for changes in the requirements should be used so as to
keep the time between user feedback sessions and input from stakeholders low.\\

R8 & Changes in dataset form at during development & It is not possible to
entirely avoid this risk due its nature and the dependence on third parties, but
steps can be taken to prevent any delays that this would cause, chiefly, a well
abstracted data parsing component which can be switched and modified if
necessary to accommodate for a new dataset format.\\

R9 & Unable to obtain required resources & Since the project does not require
many resources, it is important to acquire these as early on in the development
process as possible, and alternative resources should be planned for, such as
local test servers.\\

R10, R11, R12 & Users not committed to the project, lack of cooperation from
  users, and users with negative attitudes toward the project & the useful of
  the finished project will depend largely on ensuring that the needs of the
  users are considered the primary goals of the design. Violating this principle
  may cause disillusionment from the people who are volunteering their time to
  assist in the project.\\

\hline
\caption{Risk mitigation strategies}
\label{tab:mitigation-strategies}
\end{longtable}
