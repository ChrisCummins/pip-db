\section{PROJECT SCHEDULE \hrulefill}
The project development is spread over a 26 week period, with 11 weeks in the
first teaching period and the remaining 15 in the second. In order to maximise
the effectiveness of this time, a list of tasks for each of the four OpenUP
development phases was constructed, and a time allowance associated with
each. The final project plan consists of 8 phases: the inception phase and
transition phases, and four iterations of elaboration and construction.  The
smaller elaboration and construction cycles were used so as to maximise the
allowance for changes in the project specification caused by user feedback and
review without causing delays in the development. This is to minimise the impact
of the ``Change in project requirements during development'' risk (R7, see page
\pageref{tab:risk-assessment}). Once the list of tasks was assembled, a Gantt
chart (Appendix \ref{appendix:project-gantt-chart}) was constructed which
ordered each of these tasks and distributed them across the timespan. Careful
ordering of the tasks ensured that there is the least chance for blocking
between activities, where one task runs over the specified time allowance and
causes later tasks to be postponed until it's finished.  The final project plan
allows for the maximum amount of parallel activities and development by ensuring
that there are adequate gaps between activities that depend on each other.

\subsection{Milestones}
In order to provide a running measure of success for the project, a set of
milestones were defined which track the development process from inception
through to transition and provides completion deadlines for a set of activities.
Two types of milestones are used: design and implementation.

Design milestones cover the design of the user interface, such as the ``look and
feel'' of the project, and the interaction design. Each design milestone is
preceded by a round of user testing, in which feedback and opinions can be
gathered by the project stakeholders in order to influence the next iteration of
design.

The implementation milestones cover the technical development, with each
milestone marking a set improvement in the implementation of the backend,
frontend, and controller, from the initial prototyping phase to the ``feature
complete'' endpoint. Unlike the design milestones, the implementation milestones
are less reliant on input from third parties and so are more a personal measure
of my own development; however they have a great value in providing exact dates
to complete the implementation of features by, allowing for early sub-system
testing and providing adequate time for final product validation.

For each milestone, a set of requirements has been created which can be used as
success criteria for deciding when a milestone has been achieved. The
requirements of the milestones are cumulative, meaning that requirements for the
final milestone of each type includes all of the requirements of the previous
milestones of that type. The requirements have been split into functional and
non-functional requirements, where functional requirements describe the
behaviour and functionality of the product, and non- functional requirements
describe the criteria which can be used to judge the functional behaviour.

\subsubsection{Design Milestones}

\paragraph{D1 First iteration design (week 3)} at this early stage of
development, the design should consist of a set of non-interactive ``paper
prototypes'' or static renders of the application interface, which can be used
as a rough guide for beginning to prototype the interaction design.

\begin{table}[H]
\centering
\begin{tabular}{ l l p{12cm} }
\textbf{ID} & \textbf{Type} & \textbf{Description}\\ \hline

D1.1 & Non-functional & A set of mock-ups for the design of common site pages:
search page, results page, details page (if applicable), advanced search, login
page, and upload new data page.\\

D1.2 & Non-functional & A set of interaction mock-ups for common site tasks:
searching for a record by protein name, searching records from a specific
source, searching for records in a pI range, performing an advanced search,
adding a new record, uploading a new dataset.\\

\hline
\end{tabular}
\caption{D1 milestone requirements}
\label{tab:d1-requirements}
\end{table}

\paragraph{D2 Second iteration design (week 13)} the user interaction design
should be the primary focus of this second iteration, with many of the common
tasks (searching for a result, looking up a record, etc.) being more tightly
defined.

\begin{table}[H]
\centering
\begin{tabular}{ l l p{12cm} }
\textbf{ID} & \textbf{Type} & \textbf{Description}\\ \hline

D2.1 & Functional & An interactive prototype which implements common site tasks:
logging in and out using credentials, searching for a record by protein name,
searching records from a specific source, searching for records in a pI range,
performing an advanced search, adding a new record, uploading a new dataset.\\

D2.2 & Non-functional & A set of interaction mock-ups for `edge case' or
uncommon events: an error on the server-side, performing a search which returns
no results, attempting to log in with incorrect credentials.\\

\hline
\end{tabular}
\caption{D2 milestone requirements}
\label{tab:d2-requirements}
\end{table}

\paragraph{D3 Third iteration design (week 18)} by the third iteration, the
interaction design should be complete, allowing the focus of development to be
placed on polishing the look and feel of the application and establishing a
common aesthetic style.

\begin{table}[H]
\centering
\begin{tabular}{ l l p{12cm} }
\textbf{ID} & \textbf{Type} & \textbf{Description}\\ \hline

D3.1 & Functional & An interactive website which implements the full interaction
design.\\

D3.2 & Non-Functional & A set of revised mock-ups for the aesthetic design of
all site pages.\\

\hline
\end{tabular}
\caption{D3 milestone requirements}
\label{tab:d3-requirements}
\end{table}

\paragraph{D4 Finalised design (week 24)} this last design milestone marks the
endpoint of all design changes, and can be used to review the quality and
effectiveness of the fully evolved product.

\begin{table}[H]
\centering
\begin{tabular}{ l l p{12cm} }
\textbf{ID} & \textbf{Type} & \textbf{Description}\\ \hline

D4.1 & Functional & An interactive website which implements the full aesthetics
and interaction design, providing 100\% coverage of all interactions and
scenarios described by the mock-ups.\\

\hline
\end{tabular}
\caption{D4 milestone requirements}
\label{tab:d4-requirements}
\end{table}

\subsubsection{Implementation Milestones}

\paragraph{M1 Initial prototype (week 11)} by the end of the first term,
breath-first and depth-first prototypes of the system which some of the more
common user tasks should have been implemented, although the underlying software
architecture and technologies are free to change for the production system.

\begin{table}[H]
\centering
\begin{tabular}{ l l p{12cm} }
\textbf{ID} & \textbf{Type} & \textbf{Description}\\ \hline

M1.1 & Functional & A breadth-first prototype which implements coverage for the
common site pages and tasks.\\

M1.2 & Functional & A depth-first prototype of the user accounts system and data
back-end.\\

M1.3 & Functional & The prototype should allow for potential users to interact
with a website which implements a limited subset of the final functionality,
allowing for early feedback on the design.\\

M1.4 & Non-Functional & An architectural design for the final system database.\\

M1.5 & Non-Functional & A tool to generate fake datasets and upload them to the
prototype for testing purposes.\\

\hline
\end{tabular}
\caption{M1 milestone requirements}
\label{tab:m1-requirements}
\end{table}

\paragraph{M2 Working system (week 18)} by week 18 the software architecture and
choice of technologies should have been fully realised, and the functional
backend of the majority of use-cases should have been implemented, along with
good test coverage of each.

\begin{table}[H]
\centering
\begin{tabular}{ l l p{12cm} }
\textbf{ID} & \textbf{Type} & \textbf{Description}\\ \hline

M2.1 & Non-Functional & A software design which stipulates the final decision on
which technologies will be used.\\

M2.2 & Non-Functional & An architectural design which covers the full model view
controller stack and components of each.\\

M2.3 & Non-Functional & A test harness and accompanying automated unit tests
with full coverage of the API under common states.\\

\hline
\end{tabular}
\caption{M2 milestone requirements}
\label{tab:m2-requirements}
\end{table}

\newpage
\paragraph{M3 Feature complete (week 24)} the feature complete milestone marks
the end of the development of new features. By this point, the system should be
fully functional and optimised, allowing for final stress and load testing to
take place, and for the system to be deployed.

\begin{table}[H]
\centering
\begin{tabular}{ l l p{12cm} }
\textbf{ID} & \textbf{Type} & \textbf{Description}\\ \hline

M3.1 & Functional & A secured server which is publicly accessible from a domain
name.\\

M3.2 & Functional & An optimised software stack which can serve pages within a
determined time limit.\\

M3.3 & Functional & A software architecture which can support datasets of up to
a million records.\\

M3.4 & Non-Functional & Full test coverage of the API and automated black-box
testing of the common website tasks.\\

M3.5 & Non-Functional & The source code should be available online and licensed
with an appropriate open source license.\\

M3.6 & Non-Functional & Full documentation coverage of the internal API.\\

\hline
\end{tabular}
\caption{M3 milestone requirements}
\label{tab:m3-requirements}
\end{table}
