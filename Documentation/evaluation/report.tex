\documentclass[12pt,twoside]{report}

% Document dimensions and margins:
\usepackage[a4paper,width=170mm,top=25mm,bottom=25mm,bindingoffset=6mm]{geometry}

\usepackage[utf8]{inputenc}
\usepackage{graphicx}
\usepackage{float}
\usepackage{hyperref}
\usepackage{graphicx}
\usepackage{gensymb}
\usepackage[title]{appendix}
\usepackage[dotinlabels]{titletoc}
\usepackage[nottoc,numbib]{tocbibind}
\usepackage{mathtools}
\usepackage{enumitem}
\usepackage{longtable}

\restylefloat{figure}
\renewcommand{\thefootnote}{\arabic{footnote}}
\newcommand*\wrapletters[1]{\wr@pletters#1\@nil}
\def\wr@pletters#1#2\@nil{#1\allowbreak\if&#2&\else\wr@pletters#2\@nil\fi}
\setenumerate{itemsep=0pt}

% Increase table of contents depth from standard 3 (subsection) to 4
% (subsubseciton):
\setcounter{tocdepth}{4}

% Sort references by the order in which they are cited
\usepackage[sorting=none]{biblatex}
\addbibresource{ref.bib}

% Fancy headers
\usepackage{fancyhdr}
\fancyhead{}
\fancyhead[RO,LE]{Protein Isoelectric Point Database}
\fancyfoot{}
\fancyfoot[LE,RO]{\thepage}
\fancyfoot[LO,CE]{Chapter \thechapter}
\fancyfoot[CO,RE]{Chris Cummins}
\pagestyle{fancy}


\begin{document}

\pagenumbering{roman}

%
% Title page. Original layout by Josh Cassidy. See:
%
%   https://www.sharelatex.com/blog/2013/08/09/thesis-series-pt5.html
%
\begin{titlepage}
  \begin{center}
    \vspace*{1cm}

    \Huge
    \textbf{Protein Isoelectric Point Database Usability Tests}

    \vspace{0.5cm}
    \LARGE
    EE4FYP Final Year Project

    \vspace{1.5cm}

    \textbf{Chris Cummins}

    \vfill

    MEng Electronic Engineering\\
    \& Computer Science

    \vspace{0.8cm}

    \includegraphics[width=0.4\textwidth]{assets/aston.jpeg}

    \Large
    Electronic Engineering\\
    School of Engineering and Applied Science\\
    Aston University\\
    April 2014

  \end{center}
\end{titlepage}

\tableofcontents

\pagenumbering{arabic}

\chapter{Test Script}

\section{Introduction and consent}
\begin{enumerate}
\item \textbf{Meet participant} and introduce yourself and project.
\item Explain the \textbf{purpose of the testing} and it's role within
  the academic project assessment. The tests are for the product, not
  of the participant.
\item Request \textbf{permission to record} audio and screen capture
  the testing device. The purpose of the recordings are to prevent me
  from having to transcribe everything that is said as it is
  happening.
\item Request permission for recording to be \textbf{distributed as
  evidence} of usability testing if required.
\end{enumerate}

\section{Testing introduction}
\begin{enumerate}
\item \textbf{Start recording}.
\item Ask for a brief overview of the \textbf{participant's
  background}, and what they're studying.
\item Ascertain the participant's understanding of the relevant
  \textbf{science background and terminology}.
\item Ascertain the participant's \textbf{familiarity with the
  dataset}. Any prior knowledge of the project should be stated here.
\item \textbf{Demonstrate} to the participant how to use the testing
  device.
\item \textbf{Thinking aloud} - provide a hands on demonstration of
  thinking aloud using the Aston website. Show how you would navigate
  to find out how to apply for a research degree in CS department.
\end{enumerate}

\section{Testing}
\begin{enumerate}
\item \textbf{First impressions} - bring up the website on the testing
  device and hand over controls to participant.
\item \textbf{Present task list} to the participant and explain why it
  exists. Unlike many websites, pip-db is a specific tool to be used
  to answer questions and perform specific searches, so it is helpful
  to have a set of staged questions rather than letting the user
  browse indiscriminately.
\item \textbf{Work through tasks} - Give a spoken introduction to each
  one, and ask that the participant says out loud which step they are
  working on.
\end{enumerate}

\section{Post-test}
\begin{enumerate}
\item What is your \textbf{overall impression} of the site?
\item Would you \textbf{use the website} for tasks like those you
  worked through?
\item \textbf{How likely} do you see yourself performing a task like
  those you worked through under normal conditions?
\item If you were to \textbf{change one thing} about the website, what
  would it be?
\item \textbf{Score out of 10}.
\end{enumerate}

\section{End of session}
\begin{enumerate}
\item \textbf{Thank participant} for their time and cooperation.
\item Any closing \textbf{thoughts or questions}?
\item \textbf{Stop recording}.
\end{enumerate}

\chapter{Task Scenarios}

The following four scenarios are to be completed by the testing
participant.

\section{Researching a specific protein}
\begin{enumerate}
\item You are conducting an experiment which requires you to know the
  isoelectric point (pI) of a protein called \textit{Lactoferrin}.
\item You would also like to know the range of isoelectric points
  (lowest and highest) for all proteins obtained from the same
  \textit{source} as \textit{Lactoferrin}.
\end{enumerate}

\section{Performing broad searches}
\begin{enumerate}
\item In this scenario, you would like to research \textit{Kinase}
  proteins. You would like to download a CSV file which contains all
  proteins which match the following criteria:
  \begin{itemize}
  \item They must contain \textit{Kinase} in their names.
  \item They were obtained from a \textit{Human} source.
  \item Their enzyme commission number begins with the three digits
    \textit{2.7.1}.
  \item They were discovered at a temperature greater than or equal to \textit{4\celsius}.
  \end{itemize}
\item Once you have downloaded the CSV file, open it in a spreasheet
  program and identify the one protein with a Molecular Weight of
  \textit{86,000}.
\end{enumerate}

\section{Further broad searches}
\begin{enumerate}
\item You would like to know the number of entries in the database
  which contain the word \textit{Kinase} in their names, and compare
  this to the number of entries which \textit{do not} contain the word
  \textit{Kinase}.
\item From the entries which do not contain the word \textit{Kinase},
  you would like to find the protein with the \textit{lowest}
  isoelectric point.
\item You would like to know the names of the authors of the PubMed
  article for this protein.
\end{enumerate}

\newpage
\section{Identifying proteins using FASTA sequence}
\begin{enumerate}
\item You have been supplied with the following protein sequence:
\begin{verbatim}
>sp|P02754|LACB_BOVIN Beta-lactoglobulin OS=Bos taurus GN=LGB PE=1 SV=3
MKCLLLALALTCGAQALIVTQTMKGLDIQKVAGTWYSLAMAASDISLLDAQSAPLRVYVE
ELKPTPEGDLEILLQKWENGECAQKKIIAEKTKIPAVFKIDALNENKVLVLDTDYKKYLL
FCMENSAEPEQSLACQCLVRTPEVDDEALEKFDKALKALPMHIRLSFNPTQLEEQCHI
\end{verbatim}

You would like to identify the protein that this sequence came from,
and download a CSV file containing the details of all protiens which
match this sequence with an isoelectric point within the range
\textit{5.1 - 5.2}.
\end{enumerate}

{\centering END OF SCENARIOS.\par}

\end{document}
